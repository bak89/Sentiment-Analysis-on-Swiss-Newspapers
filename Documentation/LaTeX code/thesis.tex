\documentclass[
	a4paper,					% paper format
	12pt,							% fontsize
	%twoside,					% double-sided
	openright,				% begin new chapter on right side
	notitlepage,			% use no standard title page
	parskip=half,			% set paragraph skip to half of a line
]{article}	

\usepackage[standard-baselineskips]{cmbright}

% Set up page dimension
%---------------------------------------------------------------------------
\usepackage{geometry}
\geometry{
	a4paper,
	left=28mm,
	right=28mm,
	top=30mm,
%	bottom=20mm,
	headheight=20mm,
	headsep=10mm,
	textheight=242mm,
	footskip=15mm
}
%---------------------------------------------------------------------------

\usepackage[english]{babel}
\usepackage[utf8]{inputenc}
\usepackage[T1]{fontenc} 
\usepackage{url}
\usepackage{quoting}
\usepackage{placeins}
\usepackage{enumerate}
\usepackage{array}
\usepackage{xurl}

%\usepackage{biblatex}
\usepackage[
backend=biber,
%style=ieee,
%style=unsrt,
sorting=none
]{biblatex}


\addbibresource{database/references.bib}
\DeclareLanguageMapping{english}{UKenglish}
%---------------------------------------------------------------------------	
% Test
\usepackage[breakable]{tcolorbox}
    \usepackage{parskip} % Stop auto-indenting (to mimic markdown behaviour)
    
    \usepackage{iftex}
    \ifPDFTeX
    	\usepackage[T1]{fontenc}
    	\usepackage{mathpazo}
    \else
    	\usepackage{fontspec}
    \fi

    % Basic figure setup, for now with no caption control since it's done
    % automatically by Pandoc (which extracts ![](path) syntax from Markdown).
    \usepackage{graphicx}
    % Maintain compatibility with old templates. Remove in nbconvert 6.0
    \let\Oldincludegraphics\includegraphics
    % Ensure that by default, figures have no caption (until we provide a
    % proper Figure object with a Caption API and a way to capture that
    % in the conversion process - todo).
    %\usepackage{caption}
    %\DeclareCaptionFormat{nocaption}{}
    %\captionsetup{format=nocaption,aboveskip=0pt,belowskip=0pt}

    \usepackage{float}
    \floatplacement{figure}{H} % forces figures to be placed at the correct location
    \usepackage{xcolor} % Allow colors to be defined
    \usepackage{enumerate} % Needed for markdown enumerations to work
    \usepackage{geometry} % Used to adjust the document margins
    \usepackage{amsmath} % Equations
    \usepackage{amssymb} % Equations
    \usepackage{textcomp} % defines textquotesingle
    % Hack from http://tex.stackexchange.com/a/47451/13684:
    \AtBeginDocument{%
        \def\PYZsq{\textquotesingle}% Upright quotes in Pygmentized code
    }
    \usepackage{upquote} % Upright quotes for verbatim code
    \usepackage{eurosym} % defines \euro
    %\usepackage[mathletters]{ucs} % Extended unicode (utf-8) support
    \usepackage{fancyvrb} % verbatim replacement that allows latex
    \usepackage{grffile} % extends the file name processing of package graphics 
                         % to support a larger range
    \makeatletter % fix for old versions of grffile with XeLaTeX
    \@ifpackagelater{grffile}{2019/11/01}
    {
      % Do nothing on new versions
    }
    {
      \def\Gread@@xetex#1{%
        \IfFileExists{"\Gin@base".bb}%
        {\Gread@eps{\Gin@base.bb}}%
        {\Gread@@xetex@aux#1}%
      }
    }
    \makeatother
    \usepackage[Export]{adjustbox} % Used to constrain images to a maximum size
    \adjustboxset{max size={0.9\linewidth}{0.9\paperheight}}

    % The hyperref package gives us a pdf with properly built
    % internal navigation ('pdf bookmarks' for the table of contents,
    % internal cross-reference links, web links for URLs, etc.)
    \usepackage{hyperref}
    % The default LaTeX title has an obnoxious amount of whitespace. By default,
    % titling removes some of it. It also provides customization options.
    \usepackage{titling}
    \usepackage{longtable} % longtable support required by pandoc >1.10
    \usepackage{booktabs}  % table support for pandoc > 1.12.2
    \usepackage[inline]{enumitem} % IRkernel/repr support (it uses the enumerate* environment)
    \usepackage[normalem]{ulem} % ulem is needed to support strikethroughs (\sout)
                                % normalem makes italics be italics, not underlines
    \usepackage{mathrsfs}
    

    
    % Colors for the hyperref package
    \definecolor{black}{rgb}{0,0,0}
    \definecolor{urlcolor}{rgb}{0,.145,.698}
    \definecolor{linkcolor}{rgb}{.71,0.21,0.01}
    \definecolor{citecolor}{rgb}{.12,.54,.11}
    \definecolor{bfhgrey}{rgb}{0.41,0.49,0.57}      % BFH grey

    % ANSI colors
    \definecolor{ansi-black}{HTML}{3E424D}
    \definecolor{ansi-black-intense}{HTML}{282C36}
    \definecolor{ansi-red}{HTML}{E75C58}
    \definecolor{ansi-red-intense}{HTML}{B22B31}
    \definecolor{ansi-green}{HTML}{00A250}
    \definecolor{ansi-green-intense}{HTML}{007427}
    \definecolor{ansi-yellow}{HTML}{DDB62B}
    \definecolor{ansi-yellow-intense}{HTML}{B27D12}
    \definecolor{ansi-blue}{HTML}{208FFB}
    \definecolor{ansi-blue-intense}{HTML}{0065CA}
    \definecolor{ansi-magenta}{HTML}{D160C4}
    \definecolor{ansi-magenta-intense}{HTML}{A03196}
    \definecolor{ansi-cyan}{HTML}{60C6C8}
    \definecolor{ansi-cyan-intense}{HTML}{258F8F}
    \definecolor{ansi-white}{HTML}{C5C1B4}
    \definecolor{ansi-white-intense}{HTML}{A1A6B2}
    \definecolor{ansi-default-inverse-fg}{HTML}{FFFFFF}
    \definecolor{ansi-default-inverse-bg}{HTML}{000000}

    % common color for the border for error outputs.
    \definecolor{outerrorbackground}{HTML}{FFDFDF}

    % commands and environments needed by pandoc snippets
    % extracted from the output of `pandoc -s`
    \providecommand{\tightlist}{%
      \setlength{\itemsep}{0pt}\setlength{\parskip}{0pt}}
    \DefineVerbatimEnvironment{Highlighting}{Verbatim}{commandchars=\\\{\}}
    % Add ',fontsize=\small' for more characters per line
    \newenvironment{Shaded}{}{}
    \newcommand{\KeywordTok}[1]{\textcolor[rgb]{0.00,0.44,0.13}{\textbf{{#1}}}}
    \newcommand{\DataTypeTok}[1]{\textcolor[rgb]{0.56,0.13,0.00}{{#1}}}
    \newcommand{\DecValTok}[1]{\textcolor[rgb]{0.25,0.63,0.44}{{#1}}}
    \newcommand{\BaseNTok}[1]{\textcolor[rgb]{0.25,0.63,0.44}{{#1}}}
    \newcommand{\FloatTok}[1]{\textcolor[rgb]{0.25,0.63,0.44}{{#1}}}
    \newcommand{\CharTok}[1]{\textcolor[rgb]{0.25,0.44,0.63}{{#1}}}
    \newcommand{\StringTok}[1]{\textcolor[rgb]{0.25,0.44,0.63}{{#1}}}
    \newcommand{\CommentTok}[1]{\textcolor[rgb]{0.38,0.63,0.69}{\textit{{#1}}}}
    \newcommand{\OtherTok}[1]{\textcolor[rgb]{0.00,0.44,0.13}{{#1}}}
    \newcommand{\AlertTok}[1]{\textcolor[rgb]{1.00,0.00,0.00}{\textbf{{#1}}}}
    \newcommand{\FunctionTok}[1]{\textcolor[rgb]{0.02,0.16,0.49}{{#1}}}
    \newcommand{\RegionMarkerTok}[1]{{#1}}
    \newcommand{\ErrorTok}[1]{\textcolor[rgb]{1.00,0.00,0.00}{\textbf{{#1}}}}
    \newcommand{\NormalTok}[1]{{#1}}
    
    % Additional commands for more recent versions of Pandoc
    \newcommand{\ConstantTok}[1]{\textcolor[rgb]{0.53,0.00,0.00}{{#1}}}
    \newcommand{\SpecialCharTok}[1]{\textcolor[rgb]{0.25,0.44,0.63}{{#1}}}
    \newcommand{\VerbatimStringTok}[1]{\textcolor[rgb]{0.25,0.44,0.63}{{#1}}}
    \newcommand{\SpecialStringTok}[1]{\textcolor[rgb]{0.73,0.40,0.53}{{#1}}}
    \newcommand{\ImportTok}[1]{{#1}}
    \newcommand{\DocumentationTok}[1]{\textcolor[rgb]{0.73,0.13,0.13}{\textit{{#1}}}}
    \newcommand{\AnnotationTok}[1]{\textcolor[rgb]{0.38,0.63,0.69}{\textbf{\textit{{#1}}}}}
    \newcommand{\CommentVarTok}[1]{\textcolor[rgb]{0.38,0.63,0.69}{\textbf{\textit{{#1}}}}}
    \newcommand{\VariableTok}[1]{\textcolor[rgb]{0.10,0.09,0.49}{{#1}}}
    \newcommand{\ControlFlowTok}[1]{\textcolor[rgb]{0.00,0.44,0.13}{\textbf{{#1}}}}
    \newcommand{\OperatorTok}[1]{\textcolor[rgb]{0.40,0.40,0.40}{{#1}}}
    \newcommand{\BuiltInTok}[1]{{#1}}
    \newcommand{\ExtensionTok}[1]{{#1}}
    \newcommand{\PreprocessorTok}[1]{\textcolor[rgb]{0.74,0.48,0.00}{{#1}}}
    \newcommand{\AttributeTok}[1]{\textcolor[rgb]{0.49,0.56,0.16}{{#1}}}
    \newcommand{\InformationTok}[1]{\textcolor[rgb]{0.38,0.63,0.69}{\textbf{\textit{{#1}}}}}
    \newcommand{\WarningTok}[1]{\textcolor[rgb]{0.38,0.63,0.69}{\textbf{\textit{{#1}}}}}
    
    
    % Define a nice break command that doesn't care if a line doesn't already
    % exist.
    \def\br{\hspace*{\fill} \\* }
    % Math Jax compatibility definitions
    \def\gt{>}
    \def\lt{<}
    \let\Oldtex\TeX
    \let\Oldlatex\LaTeX
    \renewcommand{\TeX}{\textrm{\Oldtex}}
    \renewcommand{\LaTeX}{\textrm{\Oldlatex}}
    % Document parameters
    % Document title
    \title{clean}
    
    
    
    
    
% Pygments definitions
\makeatletter
\def\PY@reset{\let\PY@it=\relax \let\PY@bf=\relax%
    \let\PY@ul=\relax \let\PY@tc=\relax%
    \let\PY@bc=\relax \let\PY@ff=\relax}
\def\PY@tok#1{\csname PY@tok@#1\endcsname}
\def\PY@toks#1+{\ifx\relax#1\empty\else%
    \PY@tok{#1}\expandafter\PY@toks\fi}
\def\PY@do#1{\PY@bc{\PY@tc{\PY@ul{%
    \PY@it{\PY@bf{\PY@ff{#1}}}}}}}
\def\PY#1#2{\PY@reset\PY@toks#1+\relax+\PY@do{#2}}

\expandafter\def\csname PY@tok@w\endcsname{\def\PY@tc##1{\textcolor[rgb]{0.73,0.73,0.73}{##1}}}
\expandafter\def\csname PY@tok@c\endcsname{\let\PY@it=\textit\def\PY@tc##1{\textcolor[rgb]{0.25,0.50,0.50}{##1}}}
\expandafter\def\csname PY@tok@cp\endcsname{\def\PY@tc##1{\textcolor[rgb]{0.74,0.48,0.00}{##1}}}
\expandafter\def\csname PY@tok@k\endcsname{\let\PY@bf=\textbf\def\PY@tc##1{\textcolor[rgb]{0.00,0.50,0.00}{##1}}}
\expandafter\def\csname PY@tok@kp\endcsname{\def\PY@tc##1{\textcolor[rgb]{0.00,0.50,0.00}{##1}}}
\expandafter\def\csname PY@tok@kt\endcsname{\def\PY@tc##1{\textcolor[rgb]{0.69,0.00,0.25}{##1}}}
\expandafter\def\csname PY@tok@o\endcsname{\def\PY@tc##1{\textcolor[rgb]{0.40,0.40,0.40}{##1}}}
\expandafter\def\csname PY@tok@ow\endcsname{\let\PY@bf=\textbf\def\PY@tc##1{\textcolor[rgb]{0.67,0.13,1.00}{##1}}}
\expandafter\def\csname PY@tok@nb\endcsname{\def\PY@tc##1{\textcolor[rgb]{0.00,0.50,0.00}{##1}}}
\expandafter\def\csname PY@tok@nf\endcsname{\def\PY@tc##1{\textcolor[rgb]{0.00,0.00,1.00}{##1}}}
\expandafter\def\csname PY@tok@nc\endcsname{\let\PY@bf=\textbf\def\PY@tc##1{\textcolor[rgb]{0.00,0.00,1.00}{##1}}}
\expandafter\def\csname PY@tok@nn\endcsname{\let\PY@bf=\textbf\def\PY@tc##1{\textcolor[rgb]{0.00,0.00,1.00}{##1}}}
\expandafter\def\csname PY@tok@ne\endcsname{\let\PY@bf=\textbf\def\PY@tc##1{\textcolor[rgb]{0.82,0.25,0.23}{##1}}}
\expandafter\def\csname PY@tok@nv\endcsname{\def\PY@tc##1{\textcolor[rgb]{0.10,0.09,0.49}{##1}}}
\expandafter\def\csname PY@tok@no\endcsname{\def\PY@tc##1{\textcolor[rgb]{0.53,0.00,0.00}{##1}}}
\expandafter\def\csname PY@tok@nl\endcsname{\def\PY@tc##1{\textcolor[rgb]{0.63,0.63,0.00}{##1}}}
\expandafter\def\csname PY@tok@ni\endcsname{\let\PY@bf=\textbf\def\PY@tc##1{\textcolor[rgb]{0.60,0.60,0.60}{##1}}}
\expandafter\def\csname PY@tok@na\endcsname{\def\PY@tc##1{\textcolor[rgb]{0.49,0.56,0.16}{##1}}}
\expandafter\def\csname PY@tok@nt\endcsname{\let\PY@bf=\textbf\def\PY@tc##1{\textcolor[rgb]{0.00,0.50,0.00}{##1}}}
\expandafter\def\csname PY@tok@nd\endcsname{\def\PY@tc##1{\textcolor[rgb]{0.67,0.13,1.00}{##1}}}
\expandafter\def\csname PY@tok@s\endcsname{\def\PY@tc##1{\textcolor[rgb]{0.73,0.13,0.13}{##1}}}
\expandafter\def\csname PY@tok@sd\endcsname{\let\PY@it=\textit\def\PY@tc##1{\textcolor[rgb]{0.73,0.13,0.13}{##1}}}
\expandafter\def\csname PY@tok@si\endcsname{\let\PY@bf=\textbf\def\PY@tc##1{\textcolor[rgb]{0.73,0.40,0.53}{##1}}}
\expandafter\def\csname PY@tok@se\endcsname{\let\PY@bf=\textbf\def\PY@tc##1{\textcolor[rgb]{0.73,0.40,0.13}{##1}}}
\expandafter\def\csname PY@tok@sr\endcsname{\def\PY@tc##1{\textcolor[rgb]{0.73,0.40,0.53}{##1}}}
\expandafter\def\csname PY@tok@ss\endcsname{\def\PY@tc##1{\textcolor[rgb]{0.10,0.09,0.49}{##1}}}
\expandafter\def\csname PY@tok@sx\endcsname{\def\PY@tc##1{\textcolor[rgb]{0.00,0.50,0.00}{##1}}}
\expandafter\def\csname PY@tok@m\endcsname{\def\PY@tc##1{\textcolor[rgb]{0.40,0.40,0.40}{##1}}}
\expandafter\def\csname PY@tok@gh\endcsname{\let\PY@bf=\textbf\def\PY@tc##1{\textcolor[rgb]{0.00,0.00,0.50}{##1}}}
\expandafter\def\csname PY@tok@gu\endcsname{\let\PY@bf=\textbf\def\PY@tc##1{\textcolor[rgb]{0.50,0.00,0.50}{##1}}}
\expandafter\def\csname PY@tok@gd\endcsname{\def\PY@tc##1{\textcolor[rgb]{0.63,0.00,0.00}{##1}}}
\expandafter\def\csname PY@tok@gi\endcsname{\def\PY@tc##1{\textcolor[rgb]{0.00,0.63,0.00}{##1}}}
\expandafter\def\csname PY@tok@gr\endcsname{\def\PY@tc##1{\textcolor[rgb]{1.00,0.00,0.00}{##1}}}
\expandafter\def\csname PY@tok@ge\endcsname{\let\PY@it=\textit}
\expandafter\def\csname PY@tok@gs\endcsname{\let\PY@bf=\textbf}
\expandafter\def\csname PY@tok@gp\endcsname{\let\PY@bf=\textbf\def\PY@tc##1{\textcolor[rgb]{0.00,0.00,0.50}{##1}}}
\expandafter\def\csname PY@tok@go\endcsname{\def\PY@tc##1{\textcolor[rgb]{0.53,0.53,0.53}{##1}}}
\expandafter\def\csname PY@tok@gt\endcsname{\def\PY@tc##1{\textcolor[rgb]{0.00,0.27,0.87}{##1}}}
\expandafter\def\csname PY@tok@err\endcsname{\def\PY@bc##1{\setlength{\fboxsep}{0pt}\fcolorbox[rgb]{1.00,0.00,0.00}{1,1,1}{\strut ##1}}}
\expandafter\def\csname PY@tok@kc\endcsname{\let\PY@bf=\textbf\def\PY@tc##1{\textcolor[rgb]{0.00,0.50,0.00}{##1}}}
\expandafter\def\csname PY@tok@kd\endcsname{\let\PY@bf=\textbf\def\PY@tc##1{\textcolor[rgb]{0.00,0.50,0.00}{##1}}}
\expandafter\def\csname PY@tok@kn\endcsname{\let\PY@bf=\textbf\def\PY@tc##1{\textcolor[rgb]{0.00,0.50,0.00}{##1}}}
\expandafter\def\csname PY@tok@kr\endcsname{\let\PY@bf=\textbf\def\PY@tc##1{\textcolor[rgb]{0.00,0.50,0.00}{##1}}}
\expandafter\def\csname PY@tok@bp\endcsname{\def\PY@tc##1{\textcolor[rgb]{0.00,0.50,0.00}{##1}}}
\expandafter\def\csname PY@tok@fm\endcsname{\def\PY@tc##1{\textcolor[rgb]{0.00,0.00,1.00}{##1}}}
\expandafter\def\csname PY@tok@vc\endcsname{\def\PY@tc##1{\textcolor[rgb]{0.10,0.09,0.49}{##1}}}
\expandafter\def\csname PY@tok@vg\endcsname{\def\PY@tc##1{\textcolor[rgb]{0.10,0.09,0.49}{##1}}}
\expandafter\def\csname PY@tok@vi\endcsname{\def\PY@tc##1{\textcolor[rgb]{0.10,0.09,0.49}{##1}}}
\expandafter\def\csname PY@tok@vm\endcsname{\def\PY@tc##1{\textcolor[rgb]{0.10,0.09,0.49}{##1}}}
\expandafter\def\csname PY@tok@sa\endcsname{\def\PY@tc##1{\textcolor[rgb]{0.73,0.13,0.13}{##1}}}
\expandafter\def\csname PY@tok@sb\endcsname{\def\PY@tc##1{\textcolor[rgb]{0.73,0.13,0.13}{##1}}}
\expandafter\def\csname PY@tok@sc\endcsname{\def\PY@tc##1{\textcolor[rgb]{0.73,0.13,0.13}{##1}}}
\expandafter\def\csname PY@tok@dl\endcsname{\def\PY@tc##1{\textcolor[rgb]{0.73,0.13,0.13}{##1}}}
\expandafter\def\csname PY@tok@s2\endcsname{\def\PY@tc##1{\textcolor[rgb]{0.73,0.13,0.13}{##1}}}
\expandafter\def\csname PY@tok@sh\endcsname{\def\PY@tc##1{\textcolor[rgb]{0.73,0.13,0.13}{##1}}}
\expandafter\def\csname PY@tok@s1\endcsname{\def\PY@tc##1{\textcolor[rgb]{0.73,0.13,0.13}{##1}}}
\expandafter\def\csname PY@tok@mb\endcsname{\def\PY@tc##1{\textcolor[rgb]{0.40,0.40,0.40}{##1}}}
\expandafter\def\csname PY@tok@mf\endcsname{\def\PY@tc##1{\textcolor[rgb]{0.40,0.40,0.40}{##1}}}
\expandafter\def\csname PY@tok@mh\endcsname{\def\PY@tc##1{\textcolor[rgb]{0.40,0.40,0.40}{##1}}}
\expandafter\def\csname PY@tok@mi\endcsname{\def\PY@tc##1{\textcolor[rgb]{0.40,0.40,0.40}{##1}}}
\expandafter\def\csname PY@tok@il\endcsname{\def\PY@tc##1{\textcolor[rgb]{0.40,0.40,0.40}{##1}}}
\expandafter\def\csname PY@tok@mo\endcsname{\def\PY@tc##1{\textcolor[rgb]{0.40,0.40,0.40}{##1}}}
\expandafter\def\csname PY@tok@ch\endcsname{\let\PY@it=\textit\def\PY@tc##1{\textcolor[rgb]{0.25,0.50,0.50}{##1}}}
\expandafter\def\csname PY@tok@cm\endcsname{\let\PY@it=\textit\def\PY@tc##1{\textcolor[rgb]{0.25,0.50,0.50}{##1}}}
\expandafter\def\csname PY@tok@cpf\endcsname{\let\PY@it=\textit\def\PY@tc##1{\textcolor[rgb]{0.25,0.50,0.50}{##1}}}
\expandafter\def\csname PY@tok@c1\endcsname{\let\PY@it=\textit\def\PY@tc##1{\textcolor[rgb]{0.25,0.50,0.50}{##1}}}
\expandafter\def\csname PY@tok@cs\endcsname{\let\PY@it=\textit\def\PY@tc##1{\textcolor[rgb]{0.25,0.50,0.50}{##1}}}

\def\PYZbs{\char`\\}
\def\PYZus{\char`\_}
\def\PYZob{\char`\{}
\def\PYZcb{\char`\}}
\def\PYZca{\char`\^}
\def\PYZam{\char`\&}
\def\PYZlt{\char`\<}
\def\PYZgt{\char`\>}
\def\PYZsh{\char`\#}
\def\PYZpc{\char`\%}
\def\PYZdl{\char`\$}
\def\PYZhy{\char`\-}
\def\PYZsq{\char`\'}
\def\PYZdq{\char`\"}
\def\PYZti{\char`\~}
% for compatibility with earlier versions
\def\PYZat{@}
\def\PYZlb{[}
\def\PYZrb{]}
\makeatother


    % For linebreaks inside Verbatim environment from package fancyvrb. 
    \makeatletter
        \newbox\Wrappedcontinuationbox 
        \newbox\Wrappedvisiblespacebox 
        \newcommand*\Wrappedvisiblespace {\textcolor{red}{\textvisiblespace}} 
        \newcommand*\Wrappedcontinuationsymbol {\textcolor{red}{\llap{\tiny$\m@th\hookrightarrow$}}} 
        \newcommand*\Wrappedcontinuationindent {3ex } 
        \newcommand*\Wrappedafterbreak {\kern\Wrappedcontinuationindent\copy\Wrappedcontinuationbox} 
        % Take advantage of the already applied Pygments mark-up to insert 
        % potential linebreaks for TeX processing. 
        %        {, <, #, %, $, ' and ": go to next line. 
        %        _, }, ^, &, >, - and ~: stay at end of broken line. 
        % Use of \textquotesingle for straight quote. 
        \newcommand*\Wrappedbreaksatspecials {% 
            \def\PYGZus{\discretionary{\char`\_}{\Wrappedafterbreak}{\char`\_}}% 
            \def\PYGZob{\discretionary{}{\Wrappedafterbreak\char`\{}{\char`\{}}% 
            \def\PYGZcb{\discretionary{\char`\}}{\Wrappedafterbreak}{\char`\}}}% 
            \def\PYGZca{\discretionary{\char`\^}{\Wrappedafterbreak}{\char`\^}}% 
            \def\PYGZam{\discretionary{\char`\&}{\Wrappedafterbreak}{\char`\&}}% 
            \def\PYGZlt{\discretionary{}{\Wrappedafterbreak\char`\<}{\char`\<}}% 
            \def\PYGZgt{\discretionary{\char`\>}{\Wrappedafterbreak}{\char`\>}}% 
            \def\PYGZsh{\discretionary{}{\Wrappedafterbreak\char`\#}{\char`\#}}% 
            \def\PYGZpc{\discretionary{}{\Wrappedafterbreak\char`\%}{\char`\%}}% 
            \def\PYGZdl{\discretionary{}{\Wrappedafterbreak\char`\$}{\char`\$}}% 
            \def\PYGZhy{\discretionary{\char`\-}{\Wrappedafterbreak}{\char`\-}}% 
            \def\PYGZsq{\discretionary{}{\Wrappedafterbreak\textquotesingle}{\textquotesingle}}% 
            \def\PYGZdq{\discretionary{}{\Wrappedafterbreak\char`\"}{\char`\"}}% 
            \def\PYGZti{\discretionary{\char`\~}{\Wrappedafterbreak}{\char`\~}}% 
        } 
        % Some characters . , ; ? ! / are not pygmentized. 
        % This macro makes them "active" and they will insert potential linebreaks 
        \newcommand*\Wrappedbreaksatpunct {% 
            \lccode`\~`\.\lowercase{\def~}{\discretionary{\hbox{\char`\.}}{\Wrappedafterbreak}{\hbox{\char`\.}}}% 
            \lccode`\~`\,\lowercase{\def~}{\discretionary{\hbox{\char`\,}}{\Wrappedafterbreak}{\hbox{\char`\,}}}% 
            \lccode`\~`\;\lowercase{\def~}{\discretionary{\hbox{\char`\;}}{\Wrappedafterbreak}{\hbox{\char`\;}}}% 
            \lccode`\~`\:\lowercase{\def~}{\discretionary{\hbox{\char`\:}}{\Wrappedafterbreak}{\hbox{\char`\:}}}% 
            \lccode`\~`\?\lowercase{\def~}{\discretionary{\hbox{\char`\?}}{\Wrappedafterbreak}{\hbox{\char`\?}}}% 
            \lccode`\~`\!\lowercase{\def~}{\discretionary{\hbox{\char`\!}}{\Wrappedafterbreak}{\hbox{\char`\!}}}% 
            \lccode`\~`\/\lowercase{\def~}{\discretionary{\hbox{\char`\/}}{\Wrappedafterbreak}{\hbox{\char`\/}}}% 
            \catcode`\.\active
            \catcode`\,\active 
            \catcode`\;\active
            \catcode`\:\active
            \catcode`\?\active
            \catcode`\!\active
            \catcode`\/\active 
            \lccode`\~`\~ 	
        }
    \makeatother

    \let\OriginalVerbatim=\Verbatim
    \makeatletter
    \renewcommand{\Verbatim}[1][1]{%
        %\parskip\z@skip
        \sbox\Wrappedcontinuationbox {\Wrappedcontinuationsymbol}%
        \sbox\Wrappedvisiblespacebox {\FV@SetupFont\Wrappedvisiblespace}%
        \def\FancyVerbFormatLine ##1{\hsize\linewidth
            \vtop{\raggedright\hyphenpenalty\z@\exhyphenpenalty\z@
                \doublehyphendemerits\z@\finalhyphendemerits\z@
                \strut ##1\strut}%
        }%
        % If the linebreak is at a space, the latter will be displayed as visible
        % space at end of first line, and a continuation symbol starts next line.
        % Stretch/shrink are however usually zero for typewriter font.
        \def\FV@Space {%
            \nobreak\hskip\z@ plus\fontdimen3\font minus\fontdimen4\font
            \discretionary{\copy\Wrappedvisiblespacebox}{\Wrappedafterbreak}
            {\kern\fontdimen2\font}%
        }%
        
        % Allow breaks at special characters using \PYG... macros.
        \Wrappedbreaksatspecials
        % Breaks at punctuation characters . , ; ? ! and / need catcode=\active 	
        \OriginalVerbatim[#1,codes*=\Wrappedbreaksatpunct]%
    }
    \makeatother

    % Exact colors from NB
    \definecolor{incolor}{HTML}{303F9F}
    \definecolor{outcolor}{HTML}{D84315}
    \definecolor{cellborder}{HTML}{CFCFCF}
    \definecolor{cellbackground}{HTML}{F7F7F7}
    

    % prompt
    \makeatletter
    \newcommand{\boxspacing}{\kern\kvtcb@left@rule\kern\kvtcb@boxsep}
    \makeatother
    \newcommand{\prompt}[4]{
        {\ttfamily\llap{{\color{#2}[#3]:\hspace{3pt}#4}}\vspace{-\baselineskip}}
    }
    
    % Prevent overflowing lines due to hard-to-break entities
    \sloppy 
    % Slightly bigger margins than the latex defaults
    
    \geometry{verbose,tmargin=1in,bmargin=1in,lmargin=1in,rmargin=1in}





%---------------------------------------------------------------------------
\usepackage{textcomp}		% additional symbols
\usepackage{ae}																
\usepackage{fancyhdr}		% simple manipulation of header and footer 
\usepackage{etoolbox}		% color manipulation of header and footer
\usepackage{float}			% floating objects
\usepackage{caption}		% for captions of figures and tables
\usepackage{booktabs}		% package for nicer tables
\usepackage{tocvsec2}		% provides means of controlling the sectional numbering

%---------------------------------------------------------------------------	

% Package to facilitate placement of boxes at absolute positions
%---------------------------------------------------------------------------
\usepackage[absolute]{textpos}
\setlength{\TPHorizModule}{1mm}
\setlength{\TPVertModule}{1mm}
%---------------------------------------------------------------------------		

%epigraph
\usepackage{epigraph}
\setlength\epigraphwidth{.8\textwidth}
%\setlength\epigraphrule{0.3pt}
%\setlength \epigraphwidth {\linewidth}
%\setlength \epigraphrule {0pt}
\AtBeginDocument{\renewcommand {\epigraphflush}{center}}
\renewcommand {\sourceflush} {center}
%---------------------------------------------------------------------------	

%table
%\renewcommand\arraystretch{1.2}
\usepackage{longtable}
\usepackage{multirow}
\setlength{\arrayrulewidth}{0.2mm}
\setlength{\tabcolsep}{12pt}
\renewcommand{\arraystretch}{1.5}
%---------------------------------------------------------------------------	

% Definition of Colors
%---------------------------------------------------------------------------
%\RequirePackage{color}                          % Color (not xcolor!)
%\definecolor{linkblue}{rgb}{0,0,0.8}            % Standard
%\definecolor{darkblue}{rgb}{0,0.08,0.45}        % Dark blue
%\definecolor{bfhgrey}{rgb}{0.41,0.49,0.57}      % BFH grey
%\definecolor{linkcolor}{rgb}{0,0,0.8}     			% Blue for the web- and cd-version!
%\definecolor{linkcolor}{rgb}{0,0,0}        			% Black for the print-version!
%---------------------------------------------------------------------------

% Hyperref Package (Create links in a pdf)
%---------------------------------------------------------------------------
\usepackage{hyperref}
\hypersetup{
    colorlinks=true,
    linkcolor=black,
    filecolor=magenta,      
    urlcolor=linkcolor,
}
%---------------------------------------------------------------------------

%Headers
\usepackage{graphicx}
\usepackage{color}              %this needs to be here also
\usepackage[rightcaption]{sidecap}
%\makeindex[columns=3, title=Alphabetical Index, intoc]
%---------------------------------------------------------------------------
\usepackage[acronym]{glossaries}
%\usepackage[nonumberlist]{glossaries}
\makeglossaries
\newglossaryentry{colab}
{
    name=Colab,
    description={Colaboratory, or “Colab” for short, is a product from Google Research. Colab allows anybody to write and execute arbitrary python code through the browser, and is especially well suited to machine learning, data analysis and education.}
}
\newglossaryentry{jupyter}
{
    name=Jupyter Notebook,
    description={open-source web application that allows you to create and share documents that contain live code, equations, visualizations and narrative text. Uses include: data cleaning and transformation, numerical simulation, statistical modeling, data visualization, machine learning, and much more}
}

\newglossaryentry{anaconda}
{
    name=Anaconda,
    description={Anaconda is a distribution of the Python and R programming languages for scientific computing, that aims to simplify package management and deployment. The distribution includes data-science packages suitable for Windows, Linux, and macOS}
}

\newglossaryentry{virtual machine}
{
    name=Virtual Machine,
    description={In computing, a virtual machine is an emulation of a computer system. Virtual machines are based on computer architectures and provide functionality of a physical computer. Their implementations may involve specialized hardware, software, or a combination}
}

\newglossaryentry{keras}
{
    name=Keras,
    description={Keras is an open-source software library that provides a Python interface for artificial neural networks. Keras acts as an interface for the TensorFlow library. Up until version 2.3 Keras supported multiple backends, including TensorFlow, Microsoft Cognitive Toolkit, R, Theano, and PlaidML}
}

\newglossaryentry{Scikit-learn}
{
    name=Scikit-learn,
    description={is a free software machine learning library for the Python programming language.It features various classification, regression and clustering algorithms including support vector machines, random forests, gradient boosting, k-means and DBSCAN, and is designed to interoperate with the Python numerical and scientific libraries NumPy and SciPy}
}

\newglossaryentry{kaggle}
{
    name=Kaggle,
    description={a subsidiary of Google LLC, is an online community of data scientists and machine learning practitioners}
}

\newglossaryentry{dataset}
{
    name=Dataset,
    description={A data set is a collection of data. In the case of tabular data, a data set corresponds to one or more database tables, where every column of a table represents a particular variable, and each row corresponds to a given record of the data set in question}
}


%\newacronym{}{}{}

%---------------------------------------------------------------------------
% Makeindex Package
%---------------------------------------------------------------------------
\usepackage{makeidx}                         		% To produce index
\makeindex                                    	% Index-Initialisation
%---------------------------------------------------------------------------

\begin{document}
%\settocdepth{section} %questo divide content da version
\providecommand{\heading}{Sentiment Analysis on Swiss Newspapers}		%  Insert Title of Thesis here				% Title form titel.tex
\providecommand{\versionnumber}{0.3}			%  actual vers
\providecommand{\versiondate}{\today}		%  actual date  				% Version and date version.tex 


% Set up header and footer
%---------------------------------------------------------------------------
\makeatletter
\patchcmd{\@fancyhead}{\rlap}{\color{bfhgrey}\rlap}{}{}		% new color of header
\patchcmd{\@fancyfoot}{\rlap}{\color{bfhgrey}\rlap}{}{}		% new color of footer
\makeatother

\fancyhf{}																		% clean all fields
\fancypagestyle{plain}{												% new definition of plain style	
	\fancyfoot[OR,EL]{\footnotesize \thepage} 	% footer right part --> page number
	\fancyfoot[OL,ER]{\footnotesize \heading, Version \versionnumber, \versiondate}	% footer even page left part 
}

\renewcommand{\sectionmark}[1]{\markboth{\thesection.  #1}{}}
\renewcommand{\headrulewidth}{0pt}				% no header stripline
\renewcommand{\footrulewidth}{0pt} 				% no bottom stripline

\pagestyle{plain}
%---------------------------------------------------------------------------


% Title Page and Abstract
%---------------------------------------------------------------------------
%\thispagestyle{empty}
%\include{leader/frontpage_without_picture}
%
% Project documentation template
% ===========================================================================
% This is part of the document "Project documentation template".
% Authors: brd3, kaa1
%

\begin{titlepage}


% BFH-Logo absolute placed at (28,12) on A4 and picture (16:9 or 15cm x 8.5cm)
% Actually not a realy satisfactory solution but working.
%---------------------------------------------------------------------------
\setlength{\unitlength}{1mm}
\begin{textblock}{20}[0,0](28,12)
	\includegraphics[scale=1.0]{images/BFH_Logo_B.png}
\end{textblock}

\begin{textblock}{154}(28,48)
	\begin{picture}(150,2)
		\put(0,0){\color{bfhgrey}\rule{150mm}{2mm}}
	\end{picture}
\end{textblock}

\begin{textblock}{154}[0,0](28,50)
	\includegraphics[scale=0.295]{images/front.png}			% define cover picture
\end{textblock}

\begin{textblock}{154}(28,135)
	\begin{picture}(150,2)
		\put(0,0){\color{bfhgrey}\rule{150mm}{2mm}}
	\end{picture}
\end{textblock}
\color{black}

% Institution / titel / subtitel / authors / experts:
%---------------------------------------------------------------------------
\begin{flushleft}

\vspace*{115mm}

\fontsize{26pt}{28pt}\selectfont 
\heading				\\							% Read heading from file leader/title.tex
\vspace{2mm}

\fontsize{16pt}{20pt}\selectfont\vspace{0.3em}
Project documentation 			\\				% Insert subheading
\vspace{5mm}

\fontsize{10pt}{12pt}\selectfont
\textbf{Bachelor thesis} \\		% Insert text
\vspace{3mm}

% Abstract (eingeben):
%---------------------------------------------------------------------------
\begin{textblock}{150}(28,190)
\fontsize{10pt}{12pt}\selectfont
Today's newspapers have the power to shape one's entire perspective on the world. For that reason, it is very important to be able to differentiate between newspapers that keep one in a rather pessimistic state of being, or newspapers that offer a more optimistic outlook on the world.
The idea is to create, train, adjust and improve a model so that it is fit to analyse various Swiss newspapers and to determine which papers are written with the most negative and positive attitude.
\end{textblock}

\begin{textblock}{150}(28,225)
\fontsize{10pt}{17pt}\selectfont
\begin{tabbing}
xxxxxxxxxxxxxxx\=xxxxxxxxxxxxxxxxxxxxxxxxxxxxxxxxxxxxxxxxxxxxxxx \kill
Degree course:	\> BSC Computer Science	\\		% insert name of degree course
Author:		\> Giorgio Bakhiet Derias		\\					% insert names
Tutor:	\> Prof. Dr. Mascha Kurpicz-Briki		\\							% insert names
Expert:		\>  Andreas D\"ursteler				\\							% insert names
Date:			\> \versiondate					\\							% read from file leader/version.tex
\end{tabbing}

\end{textblock}
\end{flushleft}

\begin{textblock}{150}(28,280)
\noindent 
\color{bfhgrey}\fontsize{9pt}{10pt}\selectfont
Berner Fachhochschule | Haute \'ecole sp\'ecialis\'ee bernoise | Bern University of Applied Sciences
\color{black}\selectfont
\end{textblock}


\end{titlepage}

%
% ===========================================================================
% EOF
%

\setcounter{page}{2}
% Control of versions :
% -----------------------------------------------

\begin{textblock}{180}(15,130)
\color{black}
\begin{huge}
Versions
\end{huge}
\vspace{5mm}

\fontsize{10pt}{18pt}\selectfont
\begin{tabbing}
xxxxxxxxxxx\=xxxxxxxxxxxxxxx\=xxxxxxxxxxxxxx\=xxxxxxxxxxxxxxxxxxxxxxxxxxxxxxxxxxxxxxxxxxxxxxx \kill
Version	\> Date	\> Status			\> Remarks		\\
0.1 \> 22.02.2021 \> Draft      \> First draft           \\
0.2 \> 23.02.2021 \> Draft      \> Contents added        \\
0.3 \> 14.03.2021 \> Addition   \> Chapter 1 added       \\
0.4 \> 18.03.2021 \> Addition   \> Chapter 2 added       \\
0.5 \> 27.03.2021 \> Addition   \> Chapter 3 added       \\
0.6 \> 03.04.2021 \> Addition   \> Chapter 3 extended    \\
0.7 \> 05.04.2021 \> Addition   \> Chapter 6,7 added       \\
0.8 \> 11.04.2021 \> Correction   \> Chapter 1,2,3,5 corrected       \\
0.9 \> 19.04.2021 \> Addition   \> Chapter 4,5 added       \\
1.0 \> 10.05.2021 \> Correction   \> Chapter 4,5,6 corrected       \\
?.? \> 31.05.2021 \> Final      \> Completed             \\
\end{tabbing}

\end{textblock}





%---------------------------------------------------------------------------

% Table of contents
%---------------------------------------------------------------------------
\newpage\leavevmode\thispagestyle{plain}\newpage
\tableofcontents
\newpage\leavevmode\thispagestyle{plain}\newpage
%---------------------------------------------------------------------------

% Main part:
%---------------------------------------------------------------------------
\section[Introduction]{Introduction}
\epigraph{\centering \textit{“Things like chatbots, machine learning tools, natural language processing, or sentiment analysis are applications of artificial intelligence that may one day profoundly change how we think about and transact in travel and local experiences.”}}{Gillian Tans}

Sentiment Analysis or Opinion Mining is a way of finding out the polarity or strength of the opinion (positive or negative) that is expressed in written text.
is used in business to understand social sentiment for their brand, a particular product or service.

\subsection[In the previous episodes]{In the previous episodes}
In the last semester, my project partner and I have for the first time stepped into the world of sentiment analysis, making use of various supports such as tutorials, explanations, and lots of research.
We then decided to use all the material we had found to build something of our own, so we tried to do a project categorizing the sentiment of hotel reviews.
We wanted to split the sentiment of the reviews into 5 categories: from awful to excellent.
At the end of the project, we managed to have an accuracy of about 0.48, taking into consideration the 5 categories was a very good result.


\subsection[Project description]{Project description}
\label{main}
The thesis project I am working on this semester, will still be on sentiment analysis, this time the model created will be used to make predictions about newspaper news, to understand the polarity of an article or the newspaper itself.
With the current rapid developments in Deep Learning, many new technologies for text analysis have emerged.
This project will make use of these technologies and develop its own "Sentiment Analysis Model".
This model can basically be imagined as a function that takes arbitrary text as input, evaluates it and classifies it into a scale of emotions from negative to positive.
To develop and train a model with this capability requires many prebuilt datasets of real examples from the Internet. These are then used to feed the model so that it can learn from these datasets and deduce correlations. Based on these correlations, the model will then be able to process arbitrary texts and assign sentiment values to them.


\subsection{Motivation}
As already mentioned, this technology allows you to crystallize negative and positive feelings from a text. Besides sentiment classification, sentiment analysis brings many other benefits in areas such as marketing and customer satisfaction. Some of these benefits could be, for example:

\begin{itemize}
\item \textbf{Customer classification}\\
Sentiment analysis allows customers to be classified according to their emotional mood. This offers the opportunity to find customers who are more willing to buy.
\item \textbf{Chatbot training}\\
With the results of the Sentiment Analysis Tool, it is possible to train chatbots to recognize and respond to specific customer sentiments.
\item \textbf{Scalability and automation}\\
As a digital tool, Sentiment Analysis can be easily extended or integrated into an automated system.
\end{itemize}

From these points, a strongly increasing tendency regarding the application of sentiment analysis can be assumed. This makes it all the more relevant for developers to get to grips with it. From the developer's perspective, learning sentiment analysis also promotes new experiences in areas such as: 

\begin{itemize}
\item Data analytics
\item Data science
\item Machine learning
\item Predictive modelling
\end{itemize}

\subsection[Goal]{Goal}
\label{main}
The core of this project is with the help of different tools like Tensorflow\cite{tensorflow}, Keras and BERT to create a "model" that takes text as input and classifies it in “positive” or “negative”. 
To calculate the polarity of the article, I will use different datasets so that I can train and test the model and see if with different datasets there different results are.
Unfortunately, due to an impediment, my partner from the previous project was not able to take part in this assignment, which means that I must continue this project alone.
For a matter of timing and difficulty, I have resized the project and the categories on which to make a prediction will no more be 5 as in the last project, but 2.
This model should be available as a single component that can be integrated into arbitrary applications.\\
For the implementation of the project, I have used several websites and tutorials:
\begin{itemize}
    \item Tutorial for Keras\cite{tutorial_keras}
    \item Deep-Learning-For-Hackers\cite{git}
    \item Sentiment Analysis with TensorFlow 2 and Keras\cite{tutorial}
    \item Deep Learning LSTM for Sentiment Analysis\cite{karikari_deep_2020}
    \item Natural Language Processing and Sentiment Analysis using Tensorflow.\cite{khan_natural_2020}
    \item Sentiment Analysis: First Steps With Python's NLTK Library \cite{python_sentiment_nodate}
    \item Sentiment Analysis using Deep Learning with Tensorflow \cite{pandey_sentiment_2020}
    \item Practical Text Classification With Python and Keras \cite{python_practical_nodate}
    \item Classify text with BERT \cite{noauthor_classify_nodate}
    \item Text Classification with BERT using Transformers for long text inputs \cite{girdhar_text_2020}
    \item Simple Transformers — Multi-Class Text Classification with BERT, RoBERTa, XLNet, XLM, and DistilBERT \cite{rajapakse_simple_2019}
    \item Fine-tuning BERT with Keras and tf.Module \cite{antyukhov_fine-tuning_2020}
\end{itemize}

\subsection{Project organization}
The table~\ref{tab:Organisation} describes who occupies which role in our project organization:
\begin{longtable}[ c ]{| m{5cm} | m{5cm}|  m{3cm}|}
 \hline
 \multicolumn{3}{| c |}{\textbf{Project organization}}\\
 \hline
 \textbf{Role in the \newline project organization} & \textbf{Name}  & \textbf{BFH\newline Abbreviation}\\
 \hline
 \endfirsthead
%
 \multicolumn{3}{c}%
 {{\bfseries Table \thetable\ continued from previous page}} \\
 \hline
 \textbf{Role in the \newline project organization} & \textbf{Name}  & \textbf{BFH Abbreviation}\\
 \hline
 \endhead
%
{Tutor}   & {Mascha Kurpicz-Briki}  & {kim3}   \\ \hline
{Expert}  & {Andreas Dürsteler}         & {-}   \\ \hline
{Developer}     & {Giorgio Bakhiet Derias}& {bakhg1}\\ \hline
                 
\caption{Project organization}
\label{tab:Organisation}\\
\end{longtable}

\subsection{Tools}
In the table~\ref{tab:Tools} you can find the different tools I used for this project:
\begin{longtable}[ c ]{| m{4cm} | m{10cm}|}
 \hline
 \multicolumn{2}{| c |}{\textbf{Tools}}\\
 \hline
 \textbf{Tool}  & \textbf{Description}\\
 \hline
 \endfirsthead
%
 \multicolumn{2}{c}%
 {{\bfseries Table \thetable\ continued from previous page}} \\
 \hline
\textbf{Tool}  & \textbf{Version}\\
 \hline
 \endhead
%
{\gls{jupyter}}           & {A document that can store code, diagrams, graphics and much more.}      \\ \hline
{Google   \gls{colab}oratory} & {Online platform to host \gls{jupyter}.} \\ \hline
{\gls{kaggle}}   & {Is an online machine learning environment and community.}      \\ \hline  
{\gls{anaconda}}          & {Application to manage libraries of larger projects.}      \\ \hline
{\gls{virtual machine}}   & {\gls{anaconda} is installed on the virtual machine to run the project.}      \\ \hline
{PyCharm}   & {IDE used for the Python language. It is developed by the company JetBrains.}      \\ \hline
{GitHub}   & {GitHub is a hosting service for software projects.}      \\ \hline
{Overleaf}   & {Overleaf is a cloud-based collaborative LaTeX editor used to write, edit, and publish scientific papers.}      \\ \hline
{MLMP}   & {MLMP is a cloud-based environment for machine learning created by BFH.}      \\ \hline
{\gls{Scikit-learn}}           & {Open source machine learning software library for the Python programming language.}      \\ \hline
{Tensorflow}   & {Open source software library for machine learning.}      \\ \hline
{\gls{keras}}             & {Tensorflow library specifically for creating neural networks.}      \\ \hline
{Ktrain}   & {is a library to help build, train, debug, and deploy neural networks in the deep learning software framework Keras.}      \\ \hline
{BERT}   & {Is a recent paper published by researchers at Google AI Language.}      \\ \hline
{Hugging Face}   & {Is an open-source provider of natural language processing (NLP) technologies.}      \\ \hline

 

\caption{Tools}
\label{tab:Tools}\\
\end{longtable}


\subsection{The development environment}
This section will explain all the components used for the development environment. 
\subsubsection{Jupyter Notebook}
The model is developed in the form of a \gls{jupyter} notebook using the Python programming language. \gls{jupyter} notebooks are documents that can contain code, text, images, diagrams and explanations. This is especially suitable for this project, because in Machine Learning it is often useful to explain the sourecode with graphs and diagrams.

\subsubsection{Google \gls{colab}oratory}
The notebook was initially hosted on Google \gls{colab}oratory\cite{colab}. This brought the following advantages:
\paragraph{Synchronization} 
By storing the notebook in a central location, Google Drive, and having everyone access the same notebook, all changes are immediately visible to all users of the notebook.
\paragraph{Project structure} 
The \gls{jupyter} allows to map the whole project in a structured way in a single file. 
\paragraph{Project dependencies}
Since the project is hosted on Google \gls{colab}, it is not necessary as a developer to install libraries and other dependencies locally.

However, as it turned out, \gls{colab} removes from the platform external resources that were imported into the project after a certain time. This meant that I had to re-import the datasets, which are essential for the development of my model, before each work. This forced me to set up a \gls{virtual machine} on which the notebook can then be run locally.

\subsubsection{The \gls{virtual machine}}
In order to have the tools installed in one place, I opted for a virtual machine.
Not only this, the VM has precisely other advantages, which are:
\begin{itemize}
    \item more computing power
    \item do not use my local machine 
    \item is always on, so it can always work
    \item the models are trained on a special machine with special software
    \item being online I can work from any location
    \item always working I can train my models also at night
    \item if by chance something goes wrong, I can backup and create another VM
\end{itemize}

\subsubsection{Kaggle}
What is Kaggle? 
As we find written in the documentation: 
\begin{quote}
    "Kaggle is an AirBnB for Data Scientists, this is where they spend their nights and weekends." – Zeeshan-ul-hassan Usmani
\end{quote}
Founded in 2010 by Anthony Goldbloom (CEO) and Ben Hamner (CTO), and acquired by Google in 2017, Kaggle enables data scientists and other developers to engage in running machine learning contests, write and share code, and to host datasets. 

Kaggle is an online community for data scientists that offers:
\begin{itemize}
    \item machine learning competitions,
    \item datasets,
    \item notebooks,
    \item access to training accelerators,
    \item education.
\end{itemize}
Thanks to its 35k datasets, Kaggle is the perfect place to start a search for a datasets, notebooks or information.

\subsubsection{Anaconda}
Anaconda is a distribution of the Python and R programming languages, used for data science, machine learning etc.
Anaconda is used to simplify package management and deployment of various libraries.
To run \gls{jupyter} with all the needed libraries, \gls{anaconda} has turned out to be a suitable solution. \gls{anaconda} is a platform that allows you to set up large projects locally by creating so-called "environments" that contain all the required configurations and libraries for the project. To my advantage, \gls{anaconda} already provides a pre-built environment for \gls{jupyter} projects, which is already equipped with all my needed tools, as you can see on image~\ref{fig:fig_01}.

\begin{figure}[ht!]
\centering
\includegraphics[width=1\textwidth]{images/anaconda.jpg}
\caption{\gls{anaconda} Navigator}
\label{fig:fig_01}
\end{figure}
\FloatBarrier

Theoretically, it would also have been possible for each developer to install \gls{anaconda} on their computer and work on the notebook, which is maintained on a Github repository. However, since the project is very hardware-heavy and compiling a machine learning model can take up to several hours, a \gls{virtual machine} turns out to be the better option.

\subsubsection{PyCharm}
is an integrated development environment (IDE) used for programming in the Python language, created by the Czech company JetBrains. I used it because it supports anaconda, git and as a debugger.
It natively supports jupyter notebooks is perfect because it is much more comfortable than jupyter itself, then it has several extensions that make programming easier and more intuitive.

\subsubsection{Github}
Provider with the function of hosting for software development and version control using Git.
Thanks to its free of charge nature it's used for most of the open-source projects.
That's why it's loved by the community of programmers, even I use it for this project to save files and use them from different locations.


\subsubsection{Overleaf}
Overleaf is cloud-based \LaTeX{} editor, which makes writing a scientific paper collaborative. \LaTeX{} is a software widely used in the academic world to create scientific texts, with useful formatting for mathematical, statistical, computer science, physics texts, which can be problematic on other text editors.
I preferred Overleaf to other solutions for the convenience, in fact there is no need to install anything, everything is accessible via the internet as with Colab.
Overleaf was useful for me to write this documentation, thanks also to its versioning nature.

\subsubsection{MLMP}
MLMP is a cloud-based environnement in which machine learning models can be trained. Similar to Colab, this environnement was created by the BFH Institute, to allow its students to be able to train models, with a more performing machine.It is possible to use it in the BFH network or through a connection to the network via VPN, the advantage lies in being able to use not only the CPU, as in normal virtual machines, but also very powerful GPUs, designed precisely for this purpose.This way I can train my model much faster, having much more computing power at my disposal.

These are the available environments:
\begin{figure}[ht!]
\centering
\includegraphics[width=0.5\textwidth]{images/bfhmlmp.jpg}
\caption{MLMP distributions}
\label{fig:fig_02}
\end{figure}
\FloatBarrier
Perchè ho scelto questo env?
differenze tra cpu e gpu, un paio di grafici e quale gpu ho usato
Un esempio di CPU vs GPU:

Come è possibile vedere la GPU ci mette molto meno tempo 

https://option40.com/blog/ml-how-much-faster-is-a-gpu


\subsubsection{Scikit-learn}
Scikit-learn is  a  free  software  machine  learning  library  for  the  Python  programming  lan-guage.It features various classification, regression and clustering algorithms includingsupport vector machines, random forests, gradient boosting, k-means and DBSCAN,and  is  designed  to  interoperate  with  the  Python  numerical  and  scientific  libraries NumPy and SciPy.

\subsubsection{Tensorflow}
Tensorflow is a machine learning framework from Google, which facilitates the process of capturing data, training models, making predictions, and refining future results.

TF is an open source library for large-scale numerical computing and machine learning, it bundles a number of machine learning and deep learning algorithms and models.
All this is provided through the python language; for the reason that it is easy to learn and implement.
The actual mathematical operations, however, are performed in high-performance c+.
\subsubsection{Keras}

\subsubsection{Ktrain}
ktrain is a lightweight wrapper for the deep learning library TensorFlow Keras (and other libraries) to help build, train, and deploy neural networks and other machine learning models. Inspired by ML framework extensions like fastai and ludwig, it is designed to make deep learning and AI more accessible and easier to apply for both newcomers and experienced practitioners.
ktrain provides support for applying many pre-trained deep learning architectures in the domain of Natural Language Processing and BERT is one of them. To solve this problem, we will be using the implementation of pre-trained BERT provided by ktrain and fine-tune it to classify whether the disaster tweets are real or not.

\subsubsection{BERT}
bert è stat of the art per fare le cose

BERT (Bidirectional Encoder Representations from Transformers) is a deep learning model developed by Google. Ever since it was open-sourced by Google, it has been adopted by many researchers and industries and has applied in solving many NLP tasks. The model has been able to achieve state of the art performance on most of the problems it has been applied upon.

9.	la difficoltà più grande è trovare la parte di bert che si addice meglio.
Infatti questo task ha bisogno di molto tempo e ricerca per poter capire quale bert faccia al caso mio.


\subsubsection{Hugging Face}
Transformers posso utilizzare perché mi da migliaia di modelli pre allenati per il taskt text classificazione, estrazione informazione, risposta a domande, ecc in più di 100 lingue.
NLP è diventato facile e accessibile per tutti.


\section{Technology and Setup}
\label{chap:tech}
This section will explain all the components used for the development environment. 
\subsection{Jupyter Notebook}
The model is developed in the form of a \gls{jupyter} \cite{noauthor_project_nodate} using the Python programming language. \gls{jupyter}s are documents that can contain code, text, images, diagrams and explanations. This is especially suitable for this project, because in Machine Learning it is often useful to explain the source code with graphs and diagrams.

\subsection{Google \gls{colab}oratory}
The notebook was initially hosted on Google \gls{colab}oratory \cite{colab}. This brought the following advantages:
\paragraph{Synchronization} 
By storing the notebook in a central location, Google Drive, and having everyone access the same notebook, all changes are immediately visible to all users of the notebook.
\paragraph{Project structure} 
The \gls{jupyter} allows to map the whole project in a structured way in a single file. 
\paragraph{Project dependencies}
Since the project is hosted on Google \gls{colab}, it is not necessary as a developer to install libraries and other dependencies locally.

However, as it turned out, \gls{colab} removes from the platform external resources that were imported into the project after a certain time. This meant that I had to re-import the datasets, which are essential for the development of my model, before each work. This forced me to set up a \gls{virtual machine} on which the notebook can then be run locally.

\subsection{The \gls{virtual machine}}
To have the tools installed in one place, I opted for a virtual machine.
Not only this, the VM has precisely other advantages, which are:
\begin{itemize}
    \item more computing power
    \item do not use my local machine 
    \item is always on, so it can always work
    \item the models are trained on a special machine with special software
    \item being online I can work from any location
    \item always working I can train my models also at night
    \item if by chance something goes wrong, I can backup and create another VM
\end{itemize}

\subsection{Kaggle}
What is Kaggle \cite{noauthor_kaggle_nodate}? 
As it is possible to find written in the documentation: 
\begin{quote}
    "Kaggle is an AirBnB for Data Scientists - this is where they spend their nights and weekends." – Zeeshan-ul-hassan Usmani \cite{zeeshan-ul-hassan_what_2018}
\end{quote}
Founded in 2010 by Anthony Goldbloom (CEO) and Ben Hamner (CTO), and acquired by Google in 2017, Kaggle enables data scientists and other developers to engage in running machine learning contests, write and share code, and to host datasets. 

Kaggle is an online community for data scientists that offers:
\begin{itemize}
    \item machine learning competitions,
    \item datasets,
    \item notebooks,
    \item access to training accelerators,
    \item education.
\end{itemize}
Thanks to its 35k datasets, Kaggle is the perfect place to start a search for a datasets, notebooks or information.

\subsection{Anaconda}
Anaconda \cite{anaconda_inc_anaconda_nodate} is a distribution of the Python and R programming languages, used for data science, machine learning etc.
Anaconda is used to simplify package management and deployment of various libraries.
To run \gls{jupyter} with all the needed libraries, \gls{anaconda} has turned out to be a suitable solution. \gls{anaconda} is a platform that allows you to set up large projects locally by creating so-called "environments" that contain all the required configurations and libraries for the project. To my advantage, \gls{anaconda} already provides a pre-built environment for \gls{jupyter} projects, which is already equipped with all my needed tools, as shown in Figure~\ref{fig:fig_01}.

\begin{figure}[ht!]
\centering
\includegraphics[width=1\textwidth]{images/anaconda.jpg}
\caption{\gls{anaconda} Navigator}
\label{fig:fig_01}
\end{figure}
\FloatBarrier

Theoretically, it would also have been possible for each developer to install \gls{anaconda} on their computer and work on the notebook, which is maintained on a Github repository. However, since the project is very hardware-heavy and compiling a machine learning model can take up to several hours, a \gls{virtual machine} turns out to be the better option.

\subsection{PyCharm}
PyCharm \cite{jetbrains_sro_pycharm_nodate} is an integrated development environment (IDE) used for programming in the Python language, created by the Czech company JetBrains. I used it because it supports anaconda, git and as a debugger.
It natively supports jupyter notebooks is perfect because it is much more comfortable than jupyter itself, then it has several extensions that make programming easier and more intuitive.

\subsection{Github}
Github \cite{github_inc_github_nodate} is a provider with the function of hosting for software development and version control using Git.
Thanks to its free of charge nature it is used for most of the open-source projects.
That is why it is loved by the community of programmers, even I use it for this project to save files and use them from different locations.


\subsection{Overleaf}
Overleaf \cite{noauthor_overleaf_nodate} is cloud-based \LaTeX{} editor, which makes writing a scientific paper collaborative. \LaTeX{} is a software widely used in the academic world to create scientific texts, with useful formatting for mathematical, statistical, computer science, physics texts, which can be problematic on other text editors.
I preferred Overleaf to other solutions for the convenience, in fact there is no need to install anything, everything is accessible via the internet as with Colab.
Overleaf was useful for me to write this documentation, thanks also to its versioning nature.

\subsection{MLMP}
MLMP \cite{berner_fachhochschule_mlmp_nodate} is a cloud-based environment in which machine learning models can be trained. Similar to Colab, this environment was created by Department of Engineering and Information Technology of the Bern University of Applied Sciences (BFH), to allow its students to be able to train models, with a more performing machine. It is possible to use it in the BFH network or through a connection to the network via VPN, the advantage lies in being able to use not only the CPU, as in normal virtual machines, but also very powerful GPUs, designed precisely for this purpose.This way I can train my model much faster, having much more computing power at my disposal.

Figure~\ref{fig:fig_02} shows the available environments on MLMP:
\begin{figure}[ht!]
\centering
\includegraphics[width=0.5\textwidth]{images/bfhmlmp.jpg}
\caption{MLMP distributions}
\label{fig:fig_02}
\end{figure}
\FloatBarrier
\textbf{Which distribution to use?}
To work I chose the distribution \textbf{mlmp-tf-pytorch} because it has the most suitable components for the work I need to do such as: Tensorflow and Anaconda in full version.

As can be seen from this blog \cite{giphy_blown_nodate} the GPU takes much less time to work than the CPU.

\subsection{Scikit-learn}
Scikit-learn \cite{noauthor_scikit-learn_nodate} is  a  free  software  machine  learning  library  for  the  Python  programming  lan-guage.It features various classification, regression and clustering algorithms includingsupport vector machines, random forests, gradient boosting, k-means and DBSCAN,and  is  designed  to  interoperate  with  the  Python  numerical  and  scientific  libraries NumPy and SciPy.

\subsection{Pandas}
Pandas \cite{noauthor_pandas_nodate} ("Python Data Analysis Library") is an open source library written in Python, for data analysis and data manipulation tasks.
Pandas can be useful for example:
\begin{itemize}
    \item Convert python lists or dictionaries, Numpy arrays, to Pandas data frames.
    \item Inspecting data frames with a lot of functions.
    \item Data manipulation like filter, sort or group by and also data cleaning.
    \item Import local datasets that can be in different formats like CSV, TSV, Excel, etc.
    \item Access remote files or CSV databases or even websites in JSON format or read SQL tables or database.
\end{itemize}

\subsection{Tensorflow}
Tensorflow \cite{noauthor_tensorflow_nodate} is a machine learning framework from Google, which facilitates the process of capturing data, training models, making predictions, and refining future results.

Tensorflow is an open source library for large-scale numerical computing and machine learning, it bundles a number of machine learning and deep learning algorithms and models.
All this is provided through the python language; since it is easy to learn and implement.
The actual mathematical operations, however, are performed in high-performance c+.

\subsection{Keras}
Keras \cite{noauthor_keras_nodate} is an open-source software library that provides a Python interface for artificial neural networks. Keras acts as an interface for the TensorFlow library.

A model with Keras consists of multiple layers, each of them performs a new data transmutation on the result from the layer above, at the end we will have a series of layers connected like a neural network.

Neural networks do not process raw data, like text files, encoded JPEG image files, or CSV files. They process vectorized and standardized representations.

For this reason, Keras comes to our support, in fact we can use Keras for all those tasks of data loading and data preprocessing.

Not only this with Keras you can also perform many other tasks such as:
\begin{itemize}
    \item Build a model
    \item Train a model with the method fit()
    \item Evaluate a model
    \item Customize the method fit() for fine-tuning
\end{itemize}
See references for more details \cite{team_keras_nodate}

\subsection{BERT}
BERT \cite{devlin_bert_2019} (Bidirectional Encoder Representations from Transformers) is an open-source deep learning model developed by Google and stat of the art for NLP tasks.

the biggest difficulty is finding the part of BERT that suits you best.
In fact, this task needs a lot of time and research to be able to figure out which BERT is right for me.

In the list below are the different versions there are of BERT:

\begin{itemize}
    \item BERT-Base, Uncased and seven more models with trained weights released by the original BERT authors.
    \item Small BERTs have the same general architecture but fewer and/or smaller Transformer blocks, which lets you explore tradeoffs between speed, size and quality.
    \item ALBERT \cite{lan_albert_2020}: four different sizes of "A Lite BERT" that reduces model size (but not computation time) by sharing parameters between layers.
    \item BERT Experts \cite{smit_chexbert_2020}: eight models that all have the BERT-base architecture but offer a choice between different pre-training domains, to align more closely with the target task.
    \item ELECTRA \cite{clark_electra_2020} has the same architecture as BERT (in three different sizes), but gets pre-trained as a discriminator in a set-up that resembles a Generative Adversarial Network (GAN).
    \item BERT with Talking-Heads Attention \cite{shazeer_talking-heads_2020} and Gated GELU \cite{shazeer_glu_2020} [base, large] has two improvements to the core of the Transformer architecture.
\end{itemize}
    
\paragraph{How BERT works?} BERT use Transformer, that learns contextual relations between words in a text.
Transformer is divided in two different mechanism:
\begin{itemize}
    \item encoder - that reads the text input,
    \item decoder - that produces a prediction.
\end{itemize}
The detailed workings of Transformer are described in a paper by Google \cite{vaswani_attention_2017}.

Transformer encoder reads the entire sequence of words at once. 

\paragraph{BERT and Fine-tuning} BERT can be used for a wide variety of language tasks, while only adding a small layer to the core model, for classification task like sentiment analysis you only need to add a layer on top of the Transformer output.

\subsection{Ktrain}
Ktrain \cite{maiya_amaiyaktrain_2021} is a lightweight wrapper open-source for the deep learning library TensorFlow Keras, and many pre-trained deep learning architectures like BERT.
Ktrain helps to build, train, and deploy neural networks and other machine learning models. 
It is designed to make deep learning and AI more accessible and easier to apply for both newcomers and experienced practitioners.

For my task, I will use the implementation of pre-trained BERT provided by ktrain and fine-tune it to classify the sentiment of the reviews.


\subsection{Hugging Face}
The Hugging Face transformers package is a very popular Python library, that provides pretrained thousands of pre-trained models for task text classification, information extraction, question answering, etc. in more than 100 languages.
Thanks to Hugging Face NLP has become easy and accessible for everyone.
As of 2019, it also supports Tensorflow2 and not just PyTorch \cite{noauthor_pytorch_nodate} as before.
More information about Huggin Face \cite{noauthor_hugging_nodate}, the related paper \cite{wolf-etal-2020-transformers} and its source code \cite{noauthor_huggingfacetransformers_2021}.

\subsection{Plotly}
Plolty \cite{noauthor_plotly_nodate} is a company that develops tools for creating data visualizations for data analysis.
it is also possible to have these tools online for statistical data for individuals or collaborations.
It has a library of different plots \cite{noauthor_plotly_nodate-1} and it is also possible to create dashboards thanks to Plotly Dash \cite{noauthor_dash_nodate}.
Having Python compatible libraries was easy for me to import into my Jupyter Notebooks.
\section{The \gls{dataset}s}
\label{chap:dataset}
The most complicated step to start with is to search for a dataset, it is the crucial part because a wrong dataset means a wrongly trained model, so searching for the dataset is crucial and takes a lot of time.
Being in fact so important I spent several hours to research, analyze different datasets, to finally find what I needed.
Initially for my work I chose the IMDb dataset, but not being ideal I then replaced it with the Hotel Review dataset. The problem with both datasets is that they are in English, and it turn out in the project that the articles were in German. I had to search for a dataset in German, and I found the Filmstarts one.

\subsection{IMDB Review Description}
"Large Movie Review Dataset" \cite{noauthor_sentiment_nodate} is a dataset about movie reviews, consisting of 50k reviews which are divided into positive and negative reviews (no neutral).
This is a very famous and used dataset in the world of sentiment classification, other information can be found through the publication "Learning Word Vectors for Sentiment Analysis" \cite{maas-EtAl:2011:ACL-HLT2011}.

\subsubsection*{Motivation of the choice}
When we already have a categorization, we can say we have a labelled dataset.
Having an already labelled dataset, I will not have to do any particular operation, I have just to remove the data that do not interest me. Another advantage is that with tensorflow this dataset is particularly good for the methods I can use that we will see later.

\subsection{Hotel Review Description}
The dataset consists of approximately 515,000 customer reviews on over 1493 luxury hotels throughout Europe. Each review has a score ranging from 1 to 10. The dataset \cite{515k_kaggle} is hosted on \gls{kaggle} and is provided by Jiashen Liu.

\subsubsection*{Motivation of the choice}
The dataset provides an optimal set of datasets for the project. These are provided with plenty of attributes. The structure of the datasets is kept very simple and understandable. Furthermore, \gls{kaggle} has rated this \gls{dataset} with a usability score of 8.2.

\subsection{Filmstarts Description}
The Filmstarts dataset \cite{guhr_oliverguhrgerman-sentiment_2021} consists of 71,229 user written movie reviews in the German language. The dataset is a collection from the German website "filmstarts.de". The users can label their reviews in the range of 0.5 to 5 stars. With 40,049 documents the majority of the reviews in this data set are positive and only 15,610 reviews are negative.

\subsubsection*{Motivation of the choice}
The choice of this dataset fell on the German language, unfortunately there are not many datasets in this language. Fortunately having a score already made it easier to define a sentiment.

\section{Work on the datasets}
\label{chap:work on dataset}
\subsection{Filmstarts Dataset}
\subsubsection*{Dataframe structure}
The dataset is a tab-separated values (TSV) file. A TSV file is a simple text format for storing data in a tabular structure.

First, I need to import the file using pandas, without importing lines with errors:

    \begin{tcolorbox}[breakable, size=fbox, boxrule=1pt, pad at break*=1mm,colback=cellbackground, colframe=cellborder]
\begin{Verbatim}[commandchars=\\\{\},fontsize=\footnotesize]
\PY{c+c1}{\PYZsh{} Load the data using pandas}
\PY{n}{film\PYZus{}de} \PY{o}{=} \PY{n}{pd}\PY{o}{.}\PY{n}{read\PYZus{}csv}\PY{p}{(}\PY{l+s+s2}{\PYZdq{}}\PY{l+s+s2}{filmstarts.tsv}\PY{l+s+s2}{\PYZdq{}}\PY{p}{,} \PY{n}{sep} \PY{o}{=} \PY{l+s+s1}{\PYZsq{}}\PY{l+s+se}{\PYZbs{}t}\PY{l+s+s1}{\PYZsq{}}\PY{p}{,}\PY{n}{encoding}\PY{o}{=}\PY{l+s+s1}{\PYZsq{}}\PY{l+s+s1}{utf8}\PY{l+s+s1}{\PYZsq{}}\PY{p}{,} \PY{n}{error\PYZus{}bad\PYZus{}lines}\PY{o}{=}\PY{k+kc}{False}\PY{p}{,} \PY{n}{warn\PYZus{}bad\PYZus{}lines}\PY{o}{=}\PY{k+kc}{True}\PY{p}{,} \PY{n}{header}\PY{o}{=}\PY{k+kc}{None}\PY{p}{)}
\end{Verbatim}
\end{tcolorbox}

The TSV file contains 3 columns, this means that one customer rating contains 3 attributes. In Table~\ref{tab:Dataframe structure F} all attributes are explained in detail:

\begin{longtable}[ c ]{| m{5cm} | m{8cm}|}
\hline
\multicolumn{2}{|c|}{\textbf{Dataframe structure}}                                                                                                         \\ \hline
\endfirsthead
%
\multicolumn{2}{c}%
{{\bfseries  Table \thetable\ continued from previous page}} \\
\hline
\multicolumn{2}{|c|}{\textbf{Dataframe structure}}                                                                                                         \\ \hline
\endhead
%
\textbf{ 0 }    & Url of the review.\\ \hline
\textbf{ 1 }    & Score Rating.\\ \hline
\textbf{ 2 }    & Text of the review.\\ \hline

\caption{Dataframe structure}
\label{tab:Dataframe structure F}\\
\end{longtable}

\subsubsection{Create an Input and Response Dataframe}
I cleaned the dataframe to remove all the columns that I did not need.
After that I renamed the columns and reordered.
    \begin{tcolorbox}[breakable, size=fbox, boxrule=1pt, pad at break*=1mm,colback=cellbackground, colframe=cellborder]
\begin{Verbatim}[commandchars=\\\{\},fontsize=\small]
\PY{n}{film\PYZus{}de} \PY{o}{=} \PY{n}{film\PYZus{}de}\PY{o}{.}\PY{n}{rename}\PY{p}{(}\PY{n}{columns}\PY{o}{=}\PY{p}{\PYZob{}}\PY{l+m+mi}{2}\PY{p}{:} \PY{l+s+s1}{\PYZsq{}}\PY{l+s+s1}{Review}\PY{l+s+s1}{\PYZsq{}}\PY{p}{,} \PY{l+m+mi}{1}\PY{p}{:} \PY{l+s+s1}{\PYZsq{}}\PY{l+s+s1}{Score}\PY{l+s+s1}{\PYZsq{}}\PY{p}{\PYZcb{}}\PY{p}{)}
\end{Verbatim}
\end{tcolorbox}

            \begin{tcolorbox}[breakable, size=fbox, boxrule=.5pt, pad at break*=1mm, opacityfill=0]
\begin{Verbatim}[commandchars=\\\{\},fontsize=\footnotesize]
                                                  Review  Score
10                                    alle teile gemeint    0.0
11     ALSO:    Ich habe in meinem Leben schon viel S{\ldots}    0.0
55     der Vermarktung!     Ich frage mich allen Erns{\ldots}    1.0
58     Ey isch hab gestern Lordof the Weed gesehen ne{\ldots}    0.0
73     Also ich muss ehrlich sagen das ich es total l{\ldots}    1.0
{\ldots}                                                  {\ldots}    {\ldots}
71073  Zwei Stunden Lebenszeit vergeudet. Hier wird v{\ldots}    0.0
71078  Traumfrauen. Für viele von uns unerreichbar, e{\ldots}    1.0
71081  So einen drecks film sorry aber mir fällt leid{\ldots}    0.0

[7744 rows x 2 columns]
\end{Verbatim}
\end{tcolorbox}

Now that I have cleaned up the dataframes with what I needed, I see how they are composed, as can be seen from the chart in Figure~\ref{fig:fig_03} :

\begin{figure}[H]
\centering
\includegraphics[width=1\textwidth]{images/output_37_1.png}
\caption{Number of ratings per category}
\label{fig:fig_03}
\end{figure}
\FloatBarrier


The "Polarity" of a review is determined by the following criteria:
\begin{itemize}
\item \textbf{"0"}: up to Score 0
\item \textbf{"1"}: up to Score 5
\end{itemize}
The code for this operation is shown here below:

        \begin{tcolorbox}[breakable, size=fbox, boxrule=1pt, pad at break*=1mm,colback=cellbackground, colframe=cellborder]
\begin{Verbatim}[commandchars=\\\{\},fontsize=\footnotesize]
\PY{c+c1}{\PYZsh{} Get review type by aforementioned method}
\PY{k}{def} \PY{n+nf}{get\PYZus{}review\PYZus{}type}\PY{p}{(}\PY{n}{review\PYZus{}score}\PY{p}{)}\PY{p}{:}
    \PY{k}{if} \PY{n}{review\PYZus{}score} \PY{o}{\PYZlt{}}\PY{o}{=} \PY{l+m+mi}{0}\PY{p}{:}
        \PY{k}{return} \PY{l+m+mi}{0}
    \PY{k}{elif} \PY{n}{review\PYZus{}score} \PY{o}{\PYZgt{}}\PY{o}{=} \PY{l+m+mi}{5}\PY{p}{:}
        \PY{k}{return} \PY{l+m+mi}{1}
    \PY{k}{else}\PY{p}{:}
        \PY{k}{return} \PY{k+kc}{None}


\PY{n}{film\PYZus{}de}\PY{p}{[}\PY{l+s+s2}{\PYZdq{}}\PY{l+s+s2}{Positive}\PY{l+s+s2}{\PYZdq{}}\PY{p}{]} \PY{o}{=} \PY{n}{film\PYZus{}de}\PY{p}{[}\PY{l+s+s2}{\PYZdq{}}\PY{l+s+s2}{Score}\PY{l+s+s2}{\PYZdq{}}\PY{p}{]}\PY{o}{.}\PY{n}{apply}\PY{p}{(}
  \PY{k}{lambda} \PY{n}{x}\PY{p}{:} \PY{n}{get\PYZus{}review\PYZus{}type}\PY{p}{(}\PY{n}{x}\PY{p}{)}
\PY{p}{)}

\PY{c+c1}{\PYZsh{} Combine only the useful columns}
\PY{n}{film\PYZus{}df\PYZus{}de} \PY{o}{=} \PY{n}{film\PYZus{}de}\PY{p}{[}\PY{p}{[}\PY{l+s+s2}{\PYZdq{}}\PY{l+s+s2}{Review}\PY{l+s+s2}{\PYZdq{}}\PY{p}{,} \PY{l+s+s2}{\PYZdq{}}\PY{l+s+s2}{Positive}\PY{l+s+s2}{\PYZdq{}}\PY{p}{]}\PY{p}{]}
\end{Verbatim}
\end{tcolorbox}

Now we have the two categories 1 ("good") and 0 ("bad"). As can be seen from the chart in Figure~\ref{fig:fig_04}  the "good" category has many more values than the "bad" category.

\begin{figure}[H]
\centering
\includegraphics[width=1\textwidth]{images/output_43_1.png}
\caption{Number of ratings per category}
\label{fig:fig_04}
\end{figure}
\FloatBarrier
\subsubsection{Resample reviews}
\label{chap:resample}
In order to be able to train my model later on, I need to have the same amount of test data for each category. For this reason, I should limit the larger category to the value of the smaller one.
The code for this operation:
        \begin{tcolorbox}[breakable, size=fbox, boxrule=1pt, pad at break*=1mm,colback=cellbackground, colframe=cellborder]
\begin{Verbatim}[commandchars=\\\{\},fontsize=\footnotesize]
\PY{c+c1}{\PYZsh{} Get same number of reviews for each type}
\PY{n}{bad\PYZus{}reviews} \PY{o}{=} \PY{n}{film\PYZus{}df\PYZus{}de}\PY{p}{[}\PY{n}{film\PYZus{}df\PYZus{}de}\PY{o}{.}\PY{n}{Positive} \PY{o}{==} \PY{l+m+mi}{0}\PY{p}{]}
\PY{n}{good\PYZus{}reviews} \PY{o}{=} \PY{n}{film\PYZus{}df\PYZus{}de}\PY{p}{[}\PY{n}{film\PYZus{}df\PYZus{}de}\PY{o}{.}\PY{n}{Positive} \PY{o}{==} \PY{l+m+mi}{1}\PY{p}{]}

\PY{n}{sample\PYZus{}len} \PY{o}{=} \PY{n+nb}{len}\PY{p}{(}\PY{n}{bad\PYZus{}reviews}\PY{p}{)}

\PY{n}{bad\PYZus{}df} \PY{o}{=} \PY{n}{bad\PYZus{}reviews}
\PY{n}{good\PYZus{}df} \PY{o}{=} \PY{n}{good\PYZus{}reviews}\PY{o}{.}\PY{n}{sample}\PY{p}{(}\PY{n}{n}\PY{o}{=}\PY{n}{sample\PYZus{}len}\PY{p}{,} \PY{n}{random\PYZus{}state}\PY{o}{=}\PY{n}{RANDOM\PYZus{}SEED}\PY{p}{)}

\PY{n}{film\PYZus{}review\PYZus{}df} \PY{o}{=} \PY{n}{good\PYZus{}df}\PY{o}{.}\PY{n}{append}\PY{p}{(}\PY{n}{bad\PYZus{}df}\PY{p}{)}\PY{o}{.}\PY{n}{reset\PYZus{}index}\PY{p}{(}\PY{n}{drop}\PY{o}{=}\PY{k+kc}{True}\PY{p}{)}
\PY{n}{film\PYZus{}review\PYZus{}df}\PY{o}{.}\PY{n}{shape}
\end{Verbatim}
\end{tcolorbox}

By doing so, the data will have  the same number of entries as in Figure~\ref{fig:fig_05}.

\begin{figure}[H]
\centering
\includegraphics[width=1\textwidth]{images/output_47_0.png}
\caption{Uniform size of the categories}
\label{fig:fig_05}
\end{figure}
\FloatBarrier

\subsubsection{Preprocessing}
In this section I deal with the preparation of the data to give then to the model.
Before I start with splitting the dataset, I randomize the entire dataset by shuffling the data in a totally random way. 
The first thing to do is to split the dataframes I got after cleaning into 3 parts:
\begin{itemize}
    \item train set
    \item validation set
    \item test set
\end{itemize}

Training set is a sample of data used to fit the model, the model trains and learns from this data.\\
Validation set is used to evaluate the trained model. With the evaluation of the model, it is possible to make a fine-tune of the model and to modify the hyperparameters.\\
Test set provides the gold standard used to evaluate the model.\\

Figure~\ref{fig:fig_06} shown an example of the split of a dataset\footnote{Reference figure \url{https://towardsdatascience.com/train-validation-and-test-sets-72cb40cba9e7}}: 
\begin{figure}[H]
\centering
\includegraphics[width=1\textwidth]{images/traintestvali.jpg}
\caption{Data split}
\label{fig:fig_06}
\end{figure}
\FloatBarrier

Thanks to the library sklearn.model\_selection.train\_test\_split I can easily split my dataset in training and test. The division is done with a ratio 80\%(training)/20\%(test)

    \begin{tcolorbox}[breakable, size=fbox, boxrule=1pt, pad at break*=1mm,colback=cellbackground, colframe=cellborder]
\begin{Verbatim}[commandchars=\\\{\},fontsize=\small]
\PY{n}{train}\PY{p}{,} \PY{n}{test} \PY{o}{=} \PY{n}{train\PYZus{}test\PYZus{}split}\PY{p}{(}\PY{n}{film\PYZus{}review\PYZus{}df}\PY{p}{,}\PY{n}{test\PYZus{}size}\PY{o}{=}\PY{l+m+mf}{0.2}\PY{p}{)}
\end{Verbatim}
\end{tcolorbox}

   \begin{Verbatim}[commandchars=\\\{\},fontsize=\small]
size of training set: 5660
size of validation set: 1416
    \end{Verbatim}

Finally, in order to have data that is usable by the Ktrain library, I need to transform both sets into lists, and eliminate rows with null values.
\section{Sentiment Analysis}
Introduzione sentiment an.
You can then explain your approach in the thesis, and still describe what you have done so far with BERT, explaining that it is already working very well without further effort and comparing to your solution, and discuss the limitations. 
\subsection{Tensorflow}
Tensorflow is a machine learning framework from Google, which facilitates the process of capturing data, training models, making predictions, and refining future results.

TF is an open source library for large-scale numerical computing and machine learning, it bundles a number of machine learning and deep learning algorithms and models.
All this is provided through the python language; for the reason that it is easy to learn and implement.
The actual mathematical operations, however, are performed in high-performance c+.

\subsection{Keras}
cosa è
cosa fa
perchè?
\subsection{BERT}
BERT (Bidirectional Encoder Representations from Transformers) is a deep learning model developed by Google. Ever since it was open-sourced by Google, it has been adopted by many researchers and industries and has applied in solving many NLP tasks. The model has been able to achieve state of the art performance on most of the problems it has been applied upon.
\subsection{Ktrain}
ktrain is a lightweight wrapper for the deep learning library TensorFlow Keras (and other libraries) to help build, train, and deploy neural networks and other machine learning models. Inspired by ML framework extensions like fastai and ludwig, it is designed to make deep learning and AI more accessible and easier to apply for both newcomers and experienced practitioners.
ktrain provides support for applying many pre-trained deep learning architectures in the domain of Natural Language Processing and BERT is one of them. To solve this problem, we will be using the implementation of pre-trained BERT provided by ktrain and fine-tune it to classify whether the disaster tweets are real or not.
\subsection{Old model}
cosa è
cosa fa
perchè?
\subsection{Create}
descrizione
\subsection{Training}
cosa è
cosa fa
perchè?
\subsection{Test}
cosa è
cosa fa
perchè?
\subsection{Tuning}
cosa è
cosa fa
perchè?
\subsection{More Test}
cosa è
cosa fa
perchè?
\section{Appendix : Models created}
In addition to the documentation, I have described in this section all the models I have created over time.
From each model I learned something and realized what I could improve.
The models are listed from first to second to last to show the learning journey.

\subsection{First model: IMDb dataset}
The first model I created was with \gls{BERT}, without any framework I simply followed the official \gls{Tensorflow} tutorial.
I trained this model the IMDb dataset.
This model was purely for educational purposes, once trained the model the only work that actually there was to do was a little fine tuning.
I used this model only as an approach to get into the world of \gls{BERT}, in fact I followed the tutorial of \gls{Tensorflow} itself to get to this result.
For a thesis work I wanted to do something of my own, and not using code and tutorial that it can already find on the net, so I tried something different.
After several searches I came across \gls{Ktrain}.

\subsubsection{Lesson learned}
I learned how to configure a model with the optimizer, loss and metric.

\textbf{What is optimizer?}\\
Optimizers are used for improving speed and performance for training a specific model \cite{noauthor_tensorflow_nodate_optimizer}. 
For my model I chose Adam optimizer. Adam optimization is a stochastic gradient descent method that is based on adaptive estimation of first order and second-order moments \cite{noauthor_tfkerasoptimizersadam_nodate}.

\textbf{What is loss?}\\
 We use a loss function to determine how far the predicted values deviate from the actual values in the training data. We change the model weights to make the loss minimum, and that is what training is all about \cite{patnaik_loss_2018}.
 
\textbf{What is a metric?}\\Calculates how often predictions matches integer labels \cite{noauthor_tfkerasmetricssparsecategoricalaccuracy_nodate}.

\subsection{Second model: Ktrain - Hotel dataset}
To be able to do something of my own I thought to work also on the dataset, so with \gls{Ktrain} I did not use the IMDB dataset, but I opted for the Hotel Reviews dataset.

\subsubsection{Hotel Review Dataset}
\paragraph*{Dataframe structure}
The CSV file contains 17 columns, this means that one customer rating contains 17 attributes. In Table~\ref{tab:Dataframe structure} all attributes are explained in detail:

\begin{longtable}[ c ]{| m{5cm} | m{8cm}|}
\hline
\multicolumn{2}{|c|}{\textbf{Dataframe structure}}                                                                                                         \\ \hline
\endfirsthead
%
\multicolumn{2}{c}%
{{\bfseries  Table \thetable\ continued from previous page}} \\
\hline
\multicolumn{2}{|c|}{\textbf{Dataframe structure}}                                                                                                         \\ \hline
\endhead
%
\textbf{Hotel\_Address  }                     & Hotel address.                                                                                  \\ \hline
\textbf{Review\_Date}                         & Date on which the customer left his comment.                                          \\ \hline
\textbf{Average\_Score}                       & Average rating. Calculation by all comments of the last year.              \\ \hline
\textbf{Hotel\_Name}                          & Hotel name.                                                                                     \\ \hline
\textbf{Reviewer\_Nationality}                & Client nationality.                                                                             \\ \hline
\textbf{Negative\_Review}                     & Negative review of the customer. If there is no negative review, it says: "No Negative". \\ \hline
\textbf{ReviewTotalNegativeWord Counts}        & Number of words in the negative review.                                                        \\ \hline
\textbf{Positive\_Review}                     & Positive review of the customer. If there is no positive review, it says: "No Positive". \\ \hline
\textbf{ReviewTotalPositiveWord Counts}        & Number of words in the positive review.                                                        \\ \hline
\textbf{Reviewer\_Score}                      & Score Rating.                                                                                \\ \hline
\textbf{TotalNumberofReviews ReviewerHasGiven} & Total number of reviews left by the customer.                                  \\ \hline
\textbf{TotalNumberof\_Reviews}               & Number of reviews of the hotel.                                                                   \\ \hline
\textbf{Tags}                                 & Tags left by the customer for the review.                                                 \\ \hline
\textbf{dayssincereview}                      & Number of days between the evaluation date and creation of the \gls{dataset}.                              \\ \hline
\textbf{AdditionalNumberof \_Scoring} & The number of reviews of the customer, which consist only of a score rating and do not include a comment. \\ \hline
\textbf{lat}                                  & Latitude of the location of the hotel.                                                                 \\ \hline
\textbf{lng}                                  & Longitude of the location of the hotel.                                                                  \\ \hline
\caption{Dataframe structure}
\label{tab:Dataframe structure}\\
\end{longtable}

\subsubsection{Process the data}
To do this first I worked on the dataset so:
\begin{itemize}
    \item cleaned up the columns I did not need,
    \item divided the dataset in positive and negative thanks to the score table, so for a value below 7 is negative and above 7 is positive,
    \item resample so that there are the same number of reviews for positive and negative.
\end{itemize}

\gls{Ktrain} provides me with a lot of useful features for my purpose.
The first \gls{Ktrain} function I used is text.texts\_from\_df.
As it is written in the documentation \cite{noauthor_amaiyaktrain_nodate}, this function allows me to:

“Loads text data from \gls{Pandas} dataframe file. Class labels are assumed to be one of the following formats:
\begin{itemize}
    \item one-hot-encoded or multi-hot-encoded arrays representing classes
    \item labels are in a single column of string or integer values representing class labels
    \item labels are a single column of numerical values for text regression.”
\end{itemize}
The advantage is that I do not have to do every single preprocessing step manually, and all the steps are followed by the library itself.

Mine is case number two, in fact my dataframe in Figure~\ref{fig:fig_24}, is composed as follows:
\begin{figure}[ht!]
\centering
\includegraphics[width=0.65\textwidth]{images/dataframe.jpg}
\caption{My resample dataframe}
\label{fig:fig_24}
\end{figure}
\FloatBarrier

In Figure~\ref{fig:fig_22} shown the function from the documentation and the code function I used:
\begin{figure}[ht!]
\centering
\includegraphics[width=0.75\textwidth]{images/textdf.jpg}
\caption{\gls{Ktrain} text function from documentation}
\label{fig:fig_22}
\end{figure}
\FloatBarrier
The function will take as feature the column "Review" and as target "Polarity", it will also automatically download an already pretrained DistilBERT model with its vocabulary.
The data for the DistilBERT model has to be preprocessed in a certain way, so I have to indicate this at the preprocess\_mode line.
    \begin{tcolorbox}[breakable, size=fbox, boxrule=1pt, pad at break*=1mm,colback=cellbackground, colframe=cellborder]

\begin{Verbatim}[commandchars=\\\{\},fontsize=\footnotesize]
\PY{n}{train}\PY{p}{,} \PY{n}{val}\PY{p}{,} \PY{n}{preproc} \PY{o}{=} \PY{n}{text}\PY{o}{.}\PY{n}{texts\PYZus{}from\PYZus{}df}\PY{p}{(}
\PY{n}{train\PYZus{}df}\PY{o}{=}\PY{n}{train}\PY{p}{,}
\PY{n}{text\PYZus{}column}\PY{o}{=}\PY{l+s+s1}{\PYZsq{}}\PY{l+s+s1}{Review}\PY{l+s+s1}{\PYZsq{}}\PY{p}{,}
\PY{n}{label\PYZus{}columns}\PY{o}{=}\PY{l+s+s1}{\PYZsq{}}\PY{l+s+s1}{Polarity}\PY{l+s+s1}{\PYZsq{}}\PY{p}{,}
\PY{n}{val\PYZus{}df}\PY{o}{=}\PY{n}{test}\PY{p}{,}
\PY{n}{maxlen}\PY{o}{=}\PY{l+m+mi}{400}\PY{p}{,}
\PY{n}{preprocess\PYZus{}mode}\PY{o}{=}\PY{l+s+s1}{\PYZsq{}}\PY{l+s+s1}{distilbert}\PY{l+s+s1}{\PYZsq{}}
\PY{p}{)}
\end{Verbatim}
\end{tcolorbox}

\begin{Verbatim}[commandchars=\\\{\},fontsize=\footnotesize]
['not\_Polarity', 'Polarity']
        not\_Polarity  Polarity
128508           1.0       0.0
78547            0.0       1.0
131921           1.0       0.0
16602            0.0       1.0
134008           1.0       0.0
['not\_Polarity', 'Polarity']
        not\_Polarity  Polarity
45323            0.0       1.0
20766            0.0       1.0
14682            0.0       1.0
145727           1.0       0.0
66400            0.0       1.0
preprocessing train{\ldots}
language: en
train sequence lengths:
        mean : 39
        95percentile : 120
        99percentile : 227
    \end{Verbatim}

    
    \begin{Verbatim}[commandchars=\\\{\},fontsize=\footnotesize]
<IPython.core.display.HTML object>
    \end{Verbatim}

    
    \begin{Verbatim}[commandchars=\\\{\},fontsize=\footnotesize]
Is Multi-Label? False
preprocessing test{\ldots}
language: en
test sequence lengths:
        mean : 40
        95percentile : 123
        99percentile : 225
    \end{Verbatim}
So with this function I load the dataframe that I processed (hotel), I use the Review column for the text to process, while I use the Polarity column to get my target.


\subsubsection{Build a Model and Wrap in Learner}
\paragraph{Classifier}
In this section is described the construction of the model, to do this first I had to see what classifier \gls{Ktrain} allows me to have:
    \begin{tcolorbox}[breakable, size=fbox, boxrule=1pt, pad at break*=1mm,colback=cellbackground, colframe=cellborder]
\begin{Verbatim}[commandchars=\\\{\},fontsize=\footnotesize]
\PY{n}{text}\PY{o}{.}\PY{n}{print\PYZus{}text\PYZus{}classifiers}\PY{p}{(}\PY{p}{)}
\end{Verbatim}
\end{tcolorbox}

    \begin{Verbatim}[commandchars=\\\{\},fontsize=\footnotesize]
fasttext: a fastText-like model [http://arxiv.org/pdf/1607.01759.pdf]
logreg: logistic regression using a trainable Embedding layer
nbsvm: NBSVM model [http://www.aclweb.org/anthology/P12-2018]
bigru: Bidirectional GRU with pretrained fasttext word vectors
[https://fasttext.cc/docs/en/crawl-vectors.html]
standard\_gru: simple 2-layer GRU with randomly initialized embeddings
bert: Bidirectional Encoder Representations from Transformers (\gls{BERT}) from
keras\_bert [https://arxiv.org/abs/1810.04805]
distilbert: distilled, smaller, and faster \gls{BERT} from Hugging Face transformers
[https://arxiv.org/abs/1910.01108]
    \end{Verbatim}

The classifier I need is distilbert. DistilBERT \cite{sanh_distilbert_2020} is a distilled version of \gls{BERT}, in fact it reduces the size of \gls{BERT} by 40\%, while retaining 97\% of its language understanding capabilities and being 60\% faster.

So it is smaller and faster implemented by \gls{Hugging Face} transformer \cite{noauthor_distilbert_nodate}.
To be able to use the text.text\_classifier function I helped myself with its documentation using some help directly in Jupyter file:
 \begin{tcolorbox}[breakable, size=fbox, boxrule=1pt, pad at break*=1mm,colback=cellbackground, colframe=cellborder]
\begin{Verbatim}[commandchars=\\\{\},fontsize=\footnotesize]
\PY{n}{help}\PY{p}{(}\PY{n}{text}\PY{o}{.}\PY{n}{text\PYZus{}classifier}\PY{p}{)}
\end{Verbatim}
\end{tcolorbox}

    \begin{Verbatim}[commandchars=\\\{\},fontsize=\footnotesize]
Help on function text\_classifier in module ktrain.text.models:

text\_classifier(name, train\_data, preproc=None, multilabel=None,
metrics=['accuracy'], verbose=1)
    Build and return a text classification model.

    Args:
        name (string): one of:
                    - 'fasttext' for FastText model
                    - 'nbsvm' for NBSVM model
                    - 'logreg' for logistic regression using embedding layers
                    - 'bigru' for Bidirectional GRU with pretrained word vectors
                    - 'bert' for \gls{BERT} Text Classification
                    - 'distilbert' for Hugging Face DistilBert model

        train\_data (tuple): a tuple of numpy.ndarrays: (x\_train, y\_train) or
ktrain.Dataset instance
                        returned from one of the texts\_from\_* functions
        preproc: a ktrain.text.TextPreprocessor instance.
                 As of v0.8.0, this is required.
        multilabel (bool):  If True, multilabel model will be returned.
                            If false, binary/multiclass model will be returned.
                            If None, multilabel will be inferred from data.
        metrics(list): metrics to use
        verbose (boolean): verbosity of output
    Return:
        model (Model): A \gls{Keras} Model instance

    \end{Verbatim}

Now that I know how it works, I can build my model so, I chose the classifier (distilbert), train data and preproc are the data I created in the previous function, putting everything together I have as shown in code:
    \begin{tcolorbox}[breakable, size=fbox, boxrule=1pt, pad at break*=1mm,colback=cellbackground, colframe=cellborder]
\begin{Verbatim}[commandchars=\\\{\},fontsize=\footnotesize]
\PY{n}{model} \PY{o}{=} \PY{n}{text}\PY{o}{.}\PY{n}{text\PYZus{}classifier}\PY{p}{(}\PY{l+s+s1}{\PYZsq{}}\PY{l+s+s1}{distilbert}\PY{l+s+s1}{\PYZsq{}}\PY{p}{,} \PY{n}{train\PYZus{}data}\PY{o}{=}\PY{n}{train}\PY{p}{,} \PY{n}{preproc}\PY{o}{=}\PY{n}{preproc}\PY{p}{)}
\end{Verbatim}
\end{tcolorbox}

\paragraph{get\_learner}
Now that I have a model, I can create a learner:
    \begin{tcolorbox}[breakable, size=fbox, boxrule=1pt, pad at break*=1mm,colback=cellbackground, colframe=cellborder]
\begin{Verbatim}[commandchars=\\\{\},fontsize=\footnotesize]
\PY{n}{learner} \PY{o}{=} \PY{n}{ktrain}\PY{o}{.}\PY{n}{get\PYZus{}learner}\PY{p}{(}\PY{n}{model}\PY{p}{,}
                             \PY{n}{train\PYZus{}data}\PY{o}{=}\PY{n}{train}\PY{p}{,} 
                             \PY{n}{val\PYZus{}data}\PY{o}{=}\PY{n}{val}\PY{p}{,} 
                             \PY{n}{batch\PYZus{}size}\PY{o}{=}\PY{l+m+mi}{6}\PY{p}{)}
\end{Verbatim}
\end{tcolorbox}

\paragraph{learner.lr\_find and plot()}
Now I simulate a workout on different learning rates and its \gls{plot} in figure~\ref{fig:fig_29}, using the functions:
    \begin{tcolorbox}[breakable, size=fbox, boxrule=1pt, pad at break*=1mm,colback=cellbackground, colframe=cellborder]
\begin{Verbatim}[commandchars=\\\{\},fontsize=\footnotesize]
\PY{n}{learner}\PY{o}{.}\PY{n}{lr\PYZus{}find}\PY{p}{(}\PY{p}{)}
\end{Verbatim}
\end{tcolorbox}

\begin{tcolorbox}[breakable, size=fbox, boxrule=1pt, pad at break*=1mm,colback=cellbackground, colframe=cellborder]
\begin{Verbatim}[commandchars=\\\{\},fontsize=\footnotesize]
\PY{n}{learner}\PY{o}{.}\PY{n}{lr\PYZus{}plot}\PY{p}{(}\PY{p}{)}
\end{Verbatim}
\end{tcolorbox}

\begin{figure}[ht!]
\centering
\includegraphics[width=1\textwidth]{images/output_97_0.png}
\caption{\gls{Ktrain} learner plot}
\label{fig:fig_29}
\end{figure}
\FloatBarrier

\paragraph{fit\_onecycle}
The fit\_onecycle method is better in speed and \gls{accuracy} as just the fit method.
The fit\_onecycle is an implementation of Leslie Smith’s 1cycle policy \cite{smith_disciplined_2018}.
By using the function:
    \begin{tcolorbox}[breakable, size=fbox, boxrule=1pt, pad at break*=1mm,colback=cellbackground, colframe=cellborder]
\begin{Verbatim}[commandchars=\\\{\},fontsize=\footnotesize]
\PY{n}{learner}\PY{o}{.}\PY{n}{fit\PYZus{}onecycle}\PY{p}{(}\PY{n}{lr} \PY{o}{=} \PY{l+m+mf}{3e\PYZhy{}5}\PY{p}{,} \PY{n}{epochs}\PY{o}{=}\PY{l+m+mi}{4}\PY{p}{)}
\end{Verbatim}
\end{tcolorbox}

 \begin{Verbatim}[commandchars=\\\{\},fontsize=\footnotesize]
begin training using onecycle policy with max lr of 3e-05{\ldots}
Epoch 1/4
23270/23270 [==============================] - 2593s 110ms/step - loss: 0.3926 -
accuracy: 0.8214 - val\_loss: 0.3282 - val\_accuracy: 0.8553
Epoch 2/4
23270/23270 [==============================] - 2587s 110ms/step - loss: 0.3142 -
accuracy: 0.8632 - val\_loss: 0.3245 - val\_accuracy: 0.8584
Epoch 3/4
23270/23270 [==============================] - 2587s 110ms/step - loss: 0.2734 -
accuracy: 0.8844 - val\_loss: 0.3210 - val\_accuracy: 0.8600
Epoch 4/4
23270/23270 [==============================] - 2588s 110ms/step - loss: 0.1708 -
accuracy: 0.9318 - val\_loss: 0.4025 - val\_accuracy: 0.8537
    \end{Verbatim}
At the 4th epoch I have about 85\% \gls{accuracy}.

\paragraph{validate}
Using the:
    \begin{tcolorbox}[breakable, size=fbox, boxrule=1pt, pad at break*=1mm,colback=cellbackground, colframe=cellborder]
\begin{Verbatim}[commandchars=\\\{\},fontsize=\footnotesize]
\PY{n}{learner}\PY{o}{.}\PY{n}{validate}\PY{p}{(}\PY{p}{)}
\end{Verbatim}
\end{tcolorbox}method I can create a \gls{Confusion matrix} on the newly trained data in order to get a more detailed picture:
    \begin{Verbatim}[commandchars=\\\{\},fontsize=\footnotesize]
              precision    recall  f1-score   support

           0       0.85      0.86      0.85     17380
           1       0.86      0.85      0.85     17525

    accuracy                           0.85     34905
   macro avg       0.85      0.85      0.85     34905
weighted avg       0.85      0.85      0.85     34905

    \end{Verbatim}
            \begin{tcolorbox}[breakable, size=fbox, boxrule=.5pt, pad at break*=1mm, opacityfill=0]
\begin{Verbatim}[commandchars=\\\{\},fontsize=\footnotesize]
array([[14883,  2497],
       [ 2610, 14915]])
\end{Verbatim}
\end{tcolorbox}
        

\paragraph{autofit}
Not satisfied with this result I tried whit autofit \cite{noauthor_amaiyaktrainautofit_nodate}.
Next, I called the function as below:
    \begin{tcolorbox}[breakable, size=fbox, boxrule=1pt, pad at break*=1mm,colback=cellbackground, colframe=cellborder]
\begin{Verbatim}[commandchars=\\\{\},fontsize=\footnotesize]
\PY{n}{learner}\PY{o}{.}\PY{n}{autofit}\PY{p}{(}\PY{l+m+mf}{3e\PYZhy{}5}\PY{p}{,} \PY{n}{reduce\PYZus{}on\PYZus{}plateau}\PY{o}{=}\PY{l+m+mi}{3}\PY{p}{,} \PY{n}{checkpoint\PYZus{}folder}\PY{o}{=}\PY{l+s+s1}{\PYZsq{}}\PY{l+s+s1}{./checkpoint/}\PY{l+s+s1}{\PYZsq{}}\PY{p}{)}
\end{Verbatim}
\end{tcolorbox}

 \begin{Verbatim}[commandchars=\\\{\},fontsize=\footnotesize]
early\_stopping automatically enabled at patience=5

begin training using triangular learning rate policy with max lr of 3e-05{\ldots}
Epoch 1/1024
23270/23270 [==============================] - 2593s 110ms/step - loss: 0.1924 -
accuracy: 0.9222 - val\_loss: 0.4034 - val\_accuracy: 0.8520
\dots
Epoch 6/1024
23270/23270 [==============================] - 2588s 110ms/step - loss: 0.0412 -
accuracy: 0.9842 - val\_loss: 0.7817 - val\_accuracy: 0.8460
Restoring model weights from the end of the best epoch.
Epoch 00006: early stopping
Weights from best epoch have been loaded into model.
    \end{Verbatim}

\subsubsection{Discuss the limitations} 
As can be seen from the autofit code, also in this case I do not exceed 85\% \gls{accuracy} on validation data, having also here a problem of overfitting.

I then wanted to see what the data was where I had the most loss, this is possible due to:learner.view\_top\_losses(n=1),\\
with the result: id:11764 | loss:8.19 | true:1 | pred:0).\\
\\
At this point I saved the model and weights so that I would have a starting point for next time.\\
I made a list of improvements I want to make in the next test.\\
In text\_from\_df function:
\begin{itemize}
    \item label\_columns = must be a list, so I have to modify this argument,
    \item label\_columns = I can try wiht two different target columns positive and negative (so edit the dataframe),
    \item val\_df = none, so the 10\% of documents in training dataframe will be used for testing/validation,
    \item maxlen = from 400 to 500,
    \item random\_state = none, to have train/test split random.
\end{itemize}

I want also to use the Interactive Training \cite{jupyter_1204}.

\subsubsection{Lesson learned}
The most important of these improvements is missing, which is the work on the dataset.
In fact, the 85\% \gls{accuracy} is due to the fact of how I made the split between positive and negative reviews.


\subsection{Third model: Ktrain - Hotel dataset optimized}
The idea for this phase is to reduce the review window, focusing only on the actual negative reviews and the very positive reviews.
The continuation for this phase is as follows:
\begin{itemize}
    \item resume the original dataset,
    \item take only the best and worst reviews, based on the "Review\_Score" column,
    \item make a resample, so that you have the two categories in equal measure,
    \item and the other improvements described in the previous chapter.
\end{itemize}

\subsubsection{Improvements}
After making the improvements described above, I was able to create a model with an \gls{accuracy} for the English language of 96\%.

\subsection{Fourth model: Automatic Ktrain - Filmstarts dataset}
\label{chap:model filmstars auto}
In Switzerland the greater part of the newspaper articles are in German language, and for this reason the models that I have trained up to this moment are not suitable to the task that they must execute, therefore I have created a new model for the German language on the base of the previous models.
Fortunately, the basis I have for creating the model is correct, what I had to do is look for a suitable dataset.
For this assignment I chose a dataset on German movies, although it was difficult to find.
For training I did exactly the same as for the previous models, so I have:
\begin{itemize}
    \item clean the dataset,
    \item I have taken in consideration for the model only the more opportune data,
    \item trained model thanks to \gls{Ktrain}.
\end{itemize}

\subsubsection{Major changes}
For this model I used as in the previous \gls{Ktrain}, with these modifications:
\begin{itemize}
    \item \gls{BERT} instead of distilBERT for greater \gls{accuracy} with the German language,
    \item automatic model,
\end{itemize}
arriving at an \gls{accuracy} of 90\%.

\subsubsection{Discuss the limitations} 
Now that I have a trained model I tried testing with a file. The file consists of 30 news articles in German.

To do this test I manually labelled each item positive/negative creating a ground true and then compared it to what the model predicted.

What I saw was that the model correctly predicted most articles, with a few issues where the article was a fact or neutral, but this is normal since I only trained for positive and negative.
Unfortunately, despite the model arrives at a good percentage (90\%) I noticed that some articles where it is clearly negative were predicted positive. 

At this point I thought the problem was with the dataset and the model so here are the improvements I want to make in the next one:
\begin{itemize}
    \item take in consideration only 0 and 5 from the dataset,
    \item modify \gls{Ktrain}.
\end{itemize}
\section{Challenges}
In diesem Abschnitt befassen wir uns mit den Herausforderungen, die sich während der Arbeit am Projekt herausgestellt haben, und wir diese behandelt haben.

\subsubsection*{Finden des geeigneten \gls{dataset}s} 
Unser erstes Problem war es, einen passenden Datensatz für unsere Bedürfnisse zu finden. Um ein Modell zu trainieren, brauchten wir einen Datensatz, der für die Sentiment-Analyse relevant, aber auch gross genug war, um ihn in Trainings- und Test-Sets aufzuteilen, ohne die Modellqualität zu verlieren. Zu unseren Gunsten hat Kaggle reichlich viele Beispiele von guten Datensätzen.

\subsubsection*{\gls{virtual machine}}
Unser zweites Problem bezieht sich auf unsere virtuelle Maschine. Diese war anfangs hardware-technisch nicht zureichend für unser Projekt ausgestattet, da unser Projekt sehr rechenintensive Programme verwendet. Nach Anfrage auf eine leistungsstärkere Maschine wurde dieses Problem gelöst. In der Zwischenzeit, sind wir wieder zu Google \gls{colab} umgezogen, dort waren bereits viele Abhängigkeiten installiert.

\subsubsection*{Google \gls{colab}}
Ein Problem, das wir mit \gls{colab} hatten, war, dass wir jedes Mal, wenn wir das Notebook öffneten, das \gls{dataset} neu importieren mussten. Dies nam jeweils eine grosse Menge an Zeit in Anspruch, da unser \gls{dataset} sehr gross ist. Wir haben auch versucht, \gls{colab} auf der virtuellen Maschine laufen zu lassen, aber \gls{colab} stoppt automatisch nach einer bestimmten Zeit die Ausführung, was nicht ideal für unsere Arbeit war, da das Trainieren eines Modells bis zu mehreren Stunden andauern kann. Letzten Endes stellte sich Anaconda als die geeignete Lösung hervor.

\subsubsection*{\gls{anaconda} und \gls{jupyter}}
Beim Übergang von \gls{colab} auf die \gls{virtual machine}, haben wir bei \gls{colab} eine Datei mit allen benötigten Libraries exportiert. Daraus haben wir auf der virtuellen Maschine mit \gls{anaconda} versucht, eine neue Umgebung zu erstellen, die alle Libraries von \gls{colab} enthält. Jedoch schlug die Ausführung mit \gls{anaconda} fehl, aufgrund eines Versionskonfliktes zwischen den Libraries. Womöglich hatte dieser Fehler einen Zusammenhang mit fehlenden Zugriffsrechten in der Systemadministration. Dementsprechend verwarfen wir die Idee eine neue Umgebung zu erstellen und arbeiteten auf der Standardumgebung von \gls{anaconda}, auf der alles ohne Probleme verlief.

\subsubsection*{Grosse Verarbeitungszeiten}
Sowohl auf \gls{colab} als auch lokal auf der virtuellen Maschine mit \gls{anaconda} ist die Verarbeitungsdauer der Datensätze mit dem Sentence Encoder mehrere Stunden lang. Um lange Wartezeiten zu umgehen, haben wir des Öfteren mit verkürzten Versionen unseres \gls{dataset}s gearbeitet. Für das finale Produkt, ist jedoch eine grosse Anzahl an Daten erforderlich, weshalb wir gegen Ende des Projektes unser Sentiment-Analysis-Model nochmals mit der ganzen Menge an Daten gefüttert haben.




\section{Conclusion}
Im jetzigen Zustand haben wir ein funktionierendes Sentiment-Analysis-Model, welches einen Text lesen und ein Sentiment dazu abgeben kann, welches sich in einem Rahmen von 5 verschiedenen Kategorien befindet. Die Genauigkeit, oder die Wahrscheinlichkeit die richtige Kategorie zu treffen, liegt dabei bei rund 60\%. Das Model ist dazu trainiert ganze Sätze zu verarbeiten. Bei einzelnen Worten könnte es deshalb etwas willkürliche Resultate ausgeben. Momentan ist das Model noch an ein Jupyter-Notebook gebunden, im nächsten Schritt jedoch, würde man dieses Model exportieren und in eine serverseitige Applikation einbinden, welche sich beispielsweise über eine API aufrufen lässt.

\section{Attachments}
In addition to the documentation, the following documents are available:
\begin{itemize}
    \item \textbf{"SentimentOnSwissNewspaper.html"}, the source code of the project in html format
    \item \textbf{"SentimentOnSwissNewspaper.pdf"}, the source code of the project in pdf format
    \item \textbf{"SentimentOnSwissNewspaper.pdf"}, the schedule of the project in pdf format
    \item \textbf{"SentimentOnSwissNewspaper.xlsx"}, the schedule of the project in excel format
    \item \textbf{"Arbeitsaufteilungein.docx"}, die Word-Datei für die Unterteilung der Kapitel
\end{itemize}


\newpage\leavevmode\thispagestyle{plain}\newpage
%---------------------------------------------------------------------------

% Declaration
%---------------------------------------------------------------------------
\clearpage
\phantomsection 
\addcontentsline{toc}{section}{Declaration of authorship}
\include{leader/declaration_authorship}
%---------------------------------------------------------------------------

% Glossary
%---------------------------------------------------------------------------
\clearpage
\phantomsection 
\addcontentsline{toc}{section}{Glossary}
\printglossary%[type=\acronymtype]
%---------------------------------------------------------------------------

% Bibliography
%---------------------------------------------------------------------------
\clearpage
\phantomsection 
\addcontentsline{toc}{section}{References}
\printbibliography
%\bibliographystyle{IEEEtranS}
%\bibliographystyle{unsrt}
%\bibliography{database/references}
% Listings
%---------------------------------------------------------------------------
\clearpage
\phantomsection 
\addcontentsline{toc}{section}{List of figures}
\listoffigures
\clearpage
\phantomsection 
\addcontentsline{toc}{section}{List of tables}
\listoftables
%---------------------------------------------------------------------------

% Index
%---------------------------------------------------------------------------
\clearpage
\phantomsection 
\addcontentsline{toc}{section}{Index}
\printindex
%---------------------------------------------------------------------------

\end{document}
