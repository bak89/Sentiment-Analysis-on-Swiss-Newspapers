\section{Conclusion}
Im jetzigen Zustand haben wir ein funktionierendes Sentiment-Analysis-Model, welches einen Text lesen und ein Sentiment dazu abgeben kann, welches sich in einem Rahmen von 5 verschiedenen Kategorien befindet. Die Genauigkeit, oder die Wahrscheinlichkeit die richtige Kategorie zu treffen, liegt dabei bei rund 60\%. Das Model ist dazu trainiert ganze Sätze zu verarbeiten. Bei einzelnen Worten könnte es deshalb etwas willkürliche Resultate ausgeben. Momentan ist das Model noch an ein Jupyter-Notebook gebunden, im nächsten Schritt jedoch, würde man dieses Model exportieren und in eine serverseitige Applikation einbinden, welche sich beispielsweise über eine API aufrufen lässt.

\section{Attachments}
In addition to the documentation, the following documents are available:
\begin{itemize}
    \item \textbf{"SentimentOnSwissNewspaper.html"}, the source code of the project in html format
    \item \textbf{"SentimentOnSwissNewspaper.pdf"}, the source code of the project in pdf format
    \item \textbf{"SentimentOnSwissNewspaper.pdf"}, the schedule of the project in pdf format
    \item \textbf{"SentimentOnSwissNewspaper.xlsx"}, the schedule of the project in excel format
    \item \textbf{"Arbeitsaufteilungein.docx"}, die Word-Datei für die Unterteilung der Kapitel
\end{itemize}

