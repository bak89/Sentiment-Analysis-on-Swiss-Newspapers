\section{Sentiment Analysis}
Introduzione sentiment an.
You can then explain your approach in the thesis, and still describe what you have done so far with BERT, explaining that it is already working very well without further effort and comparing to your solution, and discuss the limitations. 
\subsection{Tensorflow}
Tensorflow is a machine learning framework from Google, which facilitates the process of capturing data, training models, making predictions, and refining future results.

TF is an open source library for large-scale numerical computing and machine learning, it bundles a number of machine learning and deep learning algorithms and models.
All this is provided through the python language; for the reason that it is easy to learn and implement.
The actual mathematical operations, however, are performed in high-performance c+.

\subsection{Keras}
cosa è
cosa fa
perchè?
\subsection{BERT}
BERT (Bidirectional Encoder Representations from Transformers) is a deep learning model developed by Google. Ever since it was open-sourced by Google, it has been adopted by many researchers and industries and has applied in solving many NLP tasks. The model has been able to achieve state of the art performance on most of the problems it has been applied upon.
\subsection{Ktrain}
ktrain is a lightweight wrapper for the deep learning library TensorFlow Keras (and other libraries) to help build, train, and deploy neural networks and other machine learning models. Inspired by ML framework extensions like fastai and ludwig, it is designed to make deep learning and AI more accessible and easier to apply for both newcomers and experienced practitioners.
ktrain provides support for applying many pre-trained deep learning architectures in the domain of Natural Language Processing and BERT is one of them. To solve this problem, we will be using the implementation of pre-trained BERT provided by ktrain and fine-tune it to classify whether the disaster tweets are real or not.
\subsection{Old model}
cosa è
cosa fa
perchè?
\subsection{Create}
descrizione
\subsection{Training}
cosa è
cosa fa
perchè?
\subsection{Test}
cosa è
cosa fa
perchè?
\subsection{Tuning}
cosa è
cosa fa
perchè?
\subsection{More Test}
cosa è
cosa fa
perchè?